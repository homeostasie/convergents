% ----------------------------------------------------------------------
\begin{frame}
  	\frametitle{Qu'est-ce que c'est ?}

\centering
\includegraphics[width=0.3\linewidth,page=1]{fig/disque-r.pdf}

\end{frame}

% ----------------------------------------------------------------------
\begin{frame}
  	\frametitle{Disque discret}

\begin{block}{Discr�tisation de Gauss d'un disque euclidien $D(C,r)$}
\centering
\includegraphics[width=0.3\linewidth,page=2]{fig/disque-r.pdf}
\end{block}

\begin{block}{D�finition analytique �quivalente}
\[
\{ (x,y) \in \mathbb{Z}^2 | (x - C_x)^2 + (y - C_y)^2 \leq r^2 \}
\]
\end{block}

\end{frame}

% ----------------------------------------------------------------------
\begin{frame}
  	\frametitle{Trois mod�les de disques discrets}

%% \begin{itemize}
%%   \item $D^e$: centre et rayon entiers
%%   \item $D^{e^2}$: centre et carr� du rayon entiers
%%   \item $D^r$: centre et carr� du rayon rationels
%% \end{itemize}

\begin{figure}[htbp]
\centering
\subfigure[$D^e(C,r)$]{\includegraphics[width=0.3\linewidth]{fig/disque-e.pdf}}\hspace{0.03\linewidth}
\subfigure[$D^{e^2}(C,r)$]{\includegraphics[width=0.3\linewidth]{fig/disque-e2.pdf}}\hspace{0.03\linewidth}
\subfigure[$D^r(C,r)$]{\includegraphics[width=0.3\linewidth,page=2]{fig/disque-r.pdf}}
\label{fig:modeles} 
\end{figure}

\end{frame}

% ----------------------------------------------------------------------
\begin{frame}
  	\frametitle{Motivation}

\begin{block}{Comprendre la distribution spatiale des points}
\centering
\includegraphics[width=0.3\linewidth]{fig/disque-alea.pdf}\hspace{0.1\linewidth}
\includegraphics[width=0.3\linewidth,page=2]{fig/disque-r.pdf}
\end{block}

\end{frame}

% ----------------------------------------------------------------------
\begin{frame}
  	\frametitle{Remarque}

\begin{block}{Seul les points du bord nous int�resse}
\centering
\includegraphics[width=0.3\linewidth,page=1]{fig/bord-vs-interior.pdf}\hspace{0.1\linewidth}
\includegraphics[width=0.3\linewidth,page=2]{fig/bord-vs-interior.pdf}
\end{block}

\end{frame}
