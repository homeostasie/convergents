% ----------------------------------------------------------------------
\begin{frame}
  	\frametitle{Un outils}

\only<1>{
\begin{block}{$\alpha$-hull, intersection de disques g�n�ralis�s}
\begin{figure}[htbp]
\centering
\subfigure[n�gatif]{\includegraphics[width=0.4\linewidth,page=1]{fig/alpha-shape.pdf}}\hspace{0.03\linewidth}
\subfigure[positif]{\includegraphics[width=0.4\linewidth,page=3]{fig/alpha-shape.pdf}}
\end{figure}
\end{block}
}

\only<2>{
\begin{block}{$\alpha$-shape, graphe}
\begin{figure}[htbp]
\centering
\subfigure[n�gatif]{\includegraphics[width=0.4\linewidth,page=2]{fig/alpha-shape.pdf}}\hspace{0.03\linewidth}
\subfigure[positif]{\includegraphics[width=0.4\linewidth,page=4]{fig/alpha-shape.pdf}}
\end{figure}
\end{block}
}

\scriptsize
\begin{thebibliography}{alpha}

\bibitem{EKS83}
[EKS83] Edelsbrunner, H., Kirkpatrick, D., Seidel, R.
\newblock On the Shape of a Set of Points in the Plane
\newblock {\em IEEE Transactions on Information Theory}, 29(4):551--559, 1983.

\end{thebibliography}

\end{frame}

% ----------------------------------------------------------------------
\begin{frame}
  	\frametitle{Int�r�t des $\alpha$-shape, $\alpha \in [-2;1/r_{min}[$}

%Les $\alpha$-shape captent et ordonnent 
% les relations de proximit� entre les points du bord. 

\includegraphics[width=0.3\linewidth,page=1]{fig/ex-alpha-shape.pdf}\hspace{0.03\linewidth}
\includegraphics[width=0.3\linewidth,page=2]{fig/ex-alpha-shape.pdf}\hspace{0.03\linewidth}
\includegraphics[width=0.3\linewidth,page=3]{fig/ex-alpha-shape.pdf}


\includegraphics[width=0.3\linewidth,page=4]{fig/ex-alpha-shape.pdf}\hspace{0.03\linewidth}
\includegraphics[width=0.3\linewidth,page=5]{fig/ex-alpha-shape.pdf}\hspace{0.03\linewidth}
\includegraphics[width=0.3\linewidth,page=6]{fig/ex-alpha-shape.pdf}

\end{frame}

% ----------------------------------------------------------------------
%\begin{frame}
%  	\frametitle{Nombre de sommets en fonction de $\alpha$}
%
%\begin{tabular}{|c|c|c|}
%\centering
%\hline \hline
%$\alpha$               & $\sharp$       & observation        \\ \hline \hline
%$-2$                   &  $O(r)$        & trivial            \\ \hline
%$-\sqrt{2}$            &  $O(r)$        & trivial            \\ \hline
%$-\epsilon$            &  $O(r^{2/3})$  & [H98], non observ� \\ \hline
%$0$                    &  $O(r^{2/3})$  & [H98]              \\ \hline
%$1/(r_{min}-\epsilon)$ &                &                    \\ \hline
%\hline
%\end{tabular}
%
%\scriptsize
%\begin{thebibliography}{alpha}
%
%\bibitem{H98}
%[H98] Har-Peled, S.
%\newblock An output sensitive algorithm for discrete convex hulls
%\newblock {\em Computational Geometry}, 10(2):125--138, 1998.
%
%\end{thebibliography}
%
%\end{frame}

% ----------------------------------------------------------------------
\begin{frame}
  	\frametitle{Triangulation de Delaunay d'ordre 0 et n}

\begin{block}{Union des $\alpha$-shapes, Delaunay, Vorono\"{i} }
\only<1>{
\begin{figure}[htbp]
\centering
\subfigure[n�gatif]{\includegraphics[width=0.45\linewidth,page=8]{fig/CDT-alpha-shape.pdf}}\hspace{0.03\linewidth}
\subfigure[positif]{\includegraphics[width=0.45\linewidth,page=2]{fig/FDT-alpha-shape.pdf}}
\end{figure}
}
\only<2>{
\begin{figure}[htbp]
\centering
\subfigure[n�gatif]{\includegraphics[width=0.45\linewidth,page=9]{fig/CDT-alpha-shape.pdf}}\hspace{0.03\linewidth}
\subfigure[positif]{\includegraphics[width=0.45\linewidth,page=3]{fig/FDT-alpha-shape.pdf}}
\end{figure}
}
\only<3>{
\begin{figure}[htbp]
\centering
\subfigure[n�gatif]{\includegraphics[width=0.45\linewidth,page=10]{fig/CDT-alpha-shape.pdf}}\hspace{0.03\linewidth}
\subfigure[positif]{\includegraphics[width=0.45\linewidth,page=4]{fig/FDT-alpha-shape.pdf}}
\end{figure}
}
\end{block}

\end{frame}

% ----------------------------------------------------------------------
\begin{frame}
  	\frametitle{Prenons un objet simple: le motif (1er octant) }

\begin{block}{Discr�tisation plancher d'un segment de droite }
\centering
\includegraphics[width=0.25\linewidth]{fig/motif.pdf}\hspace{0.1\linewidth}
\includegraphics[width=0.3\linewidth]{fig/motifs.pdf}
%\end{block}
%
%\begin{block}{D�finition analytique �quivalente}
\[
\{ (x,y) \in \mathbb{Z}^2 | 0 \leq ax + by < (a+b), \: 0 \leq x+y \leq (a+b) \}
\]
\end{block}

\alert{Triangulations de Delaunay d'ordre 0 et n d'un motif ?}

\end{frame}
