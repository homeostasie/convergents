\section{Triangulation de Delaunay d'ordre n (FDT)}

  \begin{frame}<beamer>
    \frametitle{Plan}
    \tableofcontents[currentsection]
  \end{frame}

% ----------------------------------------------------------------------
\begin{frame}
  	\frametitle{R�sultat}

\centering
\includegraphics[width=0.4\linewidth,page=1]{fig/FDT-schema.pdf}

\begin{block}{Topologie}
Soit $n = O(\max{(a,b)})$, la profondeur de d�veloppement 
en fractions continues de $a/b$, la pente du motif. 
\scriptsize
 \begin{itemize}
  \item $n+2$ sommets
  \item $2n+1$ ar�tes ($n+2$ au bord et $n-1$ internes)
  \item $n$ faces
 \end{itemize}
\end{block}

~

%ref
\scriptsize
\begin{thebibliography}{alpha}

\bibitem{R13}
[R13] Roussillon, T.,  
\newblock Euclidean Farthest-Point Vorono\"{i} Diagram of a Pattern
\newblock {\em soumis � Discrete Applied Mathematics}, 2013.

\end{thebibliography}

\end{frame}
