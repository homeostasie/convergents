\newcommand{\vect}[1]{\ensuremath{\overrightarrow{#1}}}
\newcommand{\conj}[1]{\ensuremath{\overline{#1}}}
%
\section{Triangulation de Delaunay d'ordre n (FDT)}

  \begin{frame}<beamer>
    \frametitle{Plan}
    \tableofcontents[currentsection]
  \end{frame}

% ----------------------------------------------------------------------
\begin{frame}
  	\frametitle{R�sultat}

\centering
\includegraphics[width=0.4\linewidth,page=1]{fig/FDT-schema.pdf}\hspace{0.05\linewidth}
\includegraphics[width=0.5\linewidth]{fig/FDT-tree.pdf}
%exemple ?

\begin{block}{Topologie}
Soit $n = O(\max{(a,b)})$, la profondeur de d�veloppement 
en fractions continues de $a/b$, la pente du motif. 
\scriptsize
 \begin{itemize}
  \item $n+2$ sommets
  \item $2n+1$ ar�tes ($n+2$ au bord et $n-1$ internes)
  \item $n$ faces
 \end{itemize}
\end{block}

~

%ref
\scriptsize
\begin{thebibliography}{alpha}

\bibitem{R13}
[R13] Roussillon, T.,  
\newblock Euclidean Farthest-Point Vorono\"{i} Diagram of a Pattern
\newblock {\em soumis � Discrete Applied Mathematics}, 2013.

\end{thebibliography}

\end{frame}

% ----------------------------------------------------------------------
\begin{frame}
  	\frametitle{Exemples}

\only<1>{
\begin{figure}[htbp]
\centering
\includegraphics[width=1\linewidth]{fig/29_12.pdf}
\end{figure}
\centering
$12/29 = [0;2,2,2,2]$
}
\only<2>{
\begin{figure}[htbp]
\centering
\includegraphics[width=0.7\linewidth,page=1]{fig/8_5x2.pdf}
\end{figure}
\centering
$2.(5/8)$ o� $5/8 = [0;1,1,1,2]$
}


\end{frame}


% ----------------------------------------------------------------------
\begin{frame}
  	\frametitle{Retour sur l'algorithme d'Euclide}

%equations
Soit $\theta(z_n) = p_n/q_n = [u_0, u_1, \ldots, u_n]$
($q_n > p_n > 0$ sont des entiers premiers entre eux). 

\begin{equation}
\label{eq:rec-rem}
\begin{array}{l}
  u_0 = 0, \quad r_{-1} = q_n, \: r_0 = p_n, \\
  \forall 1 \leq k \leq n, \:
  u_k = \bigg\lfloor\frac{r_{k-2}}{r_{k-1}}\bigg\rfloor, \:
  r_k = r_{k-2} - u_k r_{k-1}. \\
\end{array}
\end{equation}

Les restes sont positifs et d�croissants.  
\begin{equation}
\label{eq:rem-order}
r_{-1} = q_n > p_n = r_{0} > \ldots > r_{n-1} = 1 > r_n = 0. 
\end{equation}


Les convergents $z_k$ sont d�finis tels que $\forall 0 \leq k \leq n$, 
$p_k/q_k$ $=$ $[u_0, u_1, \ldots, u_k]$. 

\begin{equation}
\label{eq:rec-conv}
\begin{array}{l}
  z_0 = 1, \: z_{-1} = i, \\
  \forall 1 \leq k \leq n, \:
  z_k = z_{k-2} + u_k z_{k-1}. \\
\end{array}
 \quad
%
\end{equation}

\end{frame}

% ----------------------------------------------------------------------
\begin{frame}
  	\frametitle{Interpr�tation g�om�trique de l'algorithme d'Euclide}

\only<1>{
\begin{figure}[htbp]
\centering
\includegraphics[width=0.5\linewidth,page=2]{interpretation-euclid.pdf}
\end{figure}
}
\only<2>{
\begin{figure}[htbp]
\centering
\includegraphics[width=0.5\linewidth,page=3]{interpretation-euclid.pdf}
\end{figure}
}
\only<3>{
\begin{figure}[htbp]
\centering
\includegraphics[width=0.5\linewidth,page=4]{interpretation-euclid.pdf}
\end{figure}
}
\only<4>{
\begin{figure}[htbp]
\centering
\includegraphics[width=0.5\linewidth,page=5]{interpretation-euclid.pdf}
\end{figure}
}
\only<5>{
\begin{figure}[htbp]
\centering
\includegraphics[width=0.5\linewidth,page=6]{interpretation-euclid.pdf}
\end{figure}
}
\only<6>{
\begin{figure}[htbp]
\centering
\includegraphics[width=0.5\linewidth,page=7]{interpretation-euclid.pdf}
\end{figure}
}
\only<7>{
\begin{figure}[htbp]
\centering
\includegraphics[width=0.5\linewidth,page=8]{interpretation-euclid.pdf}
\end{figure}
}
\only<8>{
\begin{figure}[htbp]
\centering
\includegraphics[width=0.5\linewidth,page=9]{interpretation-euclid.pdf}
\end{figure}
}
\only<9>{
\begin{figure}[htbp]
\centering
\includegraphics[width=0.5\linewidth,page=10]{interpretation-euclid.pdf}
\end{figure}
}

\begin{equation}
\label{eq:rem-conv}
\forall 0 \leq k \leq n, \:
r_k z_{k-1} + r_{k-1} z_k = z_n.
\end{equation}

\end{frame}

% ----------------------------------------------------------------------
\begin{frame}
  	\frametitle{Convergents et enveloppe convexe: observation}

%convergents + sommets et aretes enveloppes convexes
\begin{figure}[htbp]
\centering
\includegraphics[width=0.4\linewidth,page=1]{8_5_Zk2.pdf}\hspace{0.05\linewidth}
\includegraphics[width=0.4\linewidth,page=2]{8_5_Zk2.pdf}

\includegraphics[width=0.4\linewidth,page=3]{8_5_Zk2.pdf}\hspace{0.05\linewidth}
\includegraphics[width=0.4\linewidth,page=1]{8_5_Hk.pdf}
\end{figure}

\end{frame}

% ----------------------------------------------------------------------
\begin{frame}
  	\frametitle{Convergents et enveloppe convexe: id�e de la preuve}

\begin{figure}[htbp]
\centering
\includegraphics[width=0.4\linewidth,page=5]{8_5_Zk2.pdf}\hspace{0.05\linewidth}
\includegraphics[width=0.4\linewidth,page=2]{8_5_Hk.pdf}
\end{figure}

%convergents, cones vides / enveloppes convexes
%
\begin{equation}
\label{eq:det-k1}
\forall 0 \leq k \leq n \quad 
q_kp_{k-1} - p_{k-1}q_k = (-1)^k.
\end{equation}
%
\begin{equation}
\label{eq:det-k2}
\forall 1 \leq k \leq n \quad 
q_kp_{k-2} - p_{k-2}q_k = -u_k(-1)^k.
\end{equation}
%
\begin{equation}
\label{eq:det-rem}
\forall -1 \leq k \leq n \quad 
q_np_k - p_kq_n = r_k(-1)^k.
\end{equation}


\end{frame}

% ----------------------------------------------------------------------
\begin{frame}
  	\frametitle{D�finition de la triangulation}

%ordre
\scriptsize
$(H_0, O, Z_n)$, $(H_1, H_0, O)$, $\ldots$, 
$(H_k, H_{k-1}, H_{k-2})$, $\ldots$, 
$(H_{n-1}, H_{n-2}, H_{n-3})$. 
\normalsize

\begin{figure}
\centering
\includegraphics[width=0.4\linewidth,page=2]{fig/FDT-schema-bis.pdf}\hspace{0.05\linewidth}
\end{figure}

\begin{block}{Id�e de la preuve}
 \begin{itemize}
  \item I. c'est une triangulation
  \item II. le cercle circonscrit de chaque triangle contient tous les points
  \begin{enumerate}
   \item \alert{deux triangles cons�cutifs partagent une ar�te; le cercle circonscrit 
du second contient le sommet oppos� du premier}
   \item conclure par r�currence 
  \end{enumerate}
 \end{itemize}
\end{block}

\end{frame}

% ----------------------------------------------------------------------
\begin{frame}
  	\frametitle{Exemple ($k < n$, $k$ impair)}

\begin{block}{Trois �quivalences}
 \begin{enumerate}
  \item Le cercle passant par $H_k$, $H_{k-1}$, $H_{k-2}$ contient $H_{k-3}$.
  \item $(\vect{H_{k-2}H_{k}},\vect{H_{k-2}H_{k-1}}) < (\vect{H_{k-3}H_k},\vect{H_{k-3}H_{k-1}})$
  \item \scriptsize{
$x = (z_n - z_{k-1} - z_{k-2})\conj{z_{k-1}}$,
$y = \conj{(z_n - z_{k} - z_{k-3})}z_{k-2}$ }
\normalsize{ et $\theta(x) < \theta(y)$.} 
 \end{enumerate}
\end{block}

\centering
\includegraphics[width=0.3\linewidth]{angles.pdf}

\begin{block}{3 est vrai car: }
 \begin{itemize}
  \item (parties r�elles) $\Re(x) > \Re(y) > 0$, par (\ref{eq:rec-conv}). 
  \item (parties imaginaires) 
Par (\ref{eq:rec-conv}), (\ref{eq:rem-conv}), 
(\ref{eq:det-k1}), (\ref{eq:det-k2}):
  \begin{itemize}
   \item $\Im(x) = (r_{k-1} - 1)$.
   \item $\Im(y) = (r_{k-1} - 1)u_k + (r_k - 1)$. 
   \item D'o� $\Im(y) \geq \Im(x) > 0$, par (\ref{eq:rem-order}). 
  \end{itemize}
 \end{itemize}
\end{block}

\end{frame}

% ----------------------------------------------------------------------
\begin{frame}
  	\frametitle{Int�r�t d'une description implicite}

\begin{figure}[htbp]
\centering
\includegraphics[width=0.35\linewidth,page=1]{fig/concat.pdf}\hspace{0.1\linewidth}
\includegraphics[width=0.35\linewidth,page=2]{fig/concat.pdf}
\end{figure}

\begin{figure}[htbp]
\centering
\includegraphics[width=0.35\linewidth,page=3]{fig/concat.pdf}\hspace{0.1\linewidth}
\includegraphics[width=0.35\linewidth,page=4]{fig/concat.pdf}
\end{figure}

\end{frame}
