\documentclass{beamer}

\usefonttheme{professionalfonts}

\usepackage{times}
\usepackage[latin1]{inputenc} %franc
\usepackage[T1]{fontenc} %ang
\usepackage{xspace}
\usepackage[english,francais]{babel}

\usepackage[]{beamerthemeliris} % paquet theme, options: nogradient,nobackground


\usepackage{graphicx} %paquet graphiques
\usepackage{epsfig}
\usepackage{subfigure}
\graphicspath{{fig/}}

\usepackage{amsmath} %math
\usepackage{amssymb}

\title[GT GEODIS 13/10/2013]
{G�om�trie des motifs de droites discr�tes}
%\subtitle{}

\author[T. Roussillon]
{Tristan Roussillon}


\date{13/10/2013}

%rappel du sommaire 
%\AtBeginSection[]
%{
%  \begin{frame}<beamer>
%    \frametitle{Outline}
%    \tableofcontents[currentsection]
%  \end{frame}
%}


\begin{document}

% ----------------------------------------------------------------------
%slide de titre
\begin{frame}%[plain]
  \titlepage
  \begin{center}
  Journ�e du GT GEODIS
  \end{center}
\end{frame}

% ----------------------------------------------------------------------
%corps de la pr�sentation
% ----------------------------------------------------------------------
\begin{frame}
  	\frametitle{Un slide}


\end{frame}

% ----------------------------------------------------------------------
\begin{frame}
  	\frametitle{Un slide}


\end{frame}

% ----------------------------------------------------------------------
\begin{frame}
  	\frametitle{Un slide}


\end{frame}

% ----------------------------------------------------------------------
\begin{frame}
  	\frametitle{Un slide}


\end{frame}


% ----------------------------------------------------------------------
\begin{frame}
  	\frametitle{Conclusion}

%-CDT et FDT d'un motif

%extension � un DSS, � un DSS+2
%d�termination de la classes des segments max. qui sont arcs max. 

%partie de motifs sur un cercle

\end{frame}

\end{document}


