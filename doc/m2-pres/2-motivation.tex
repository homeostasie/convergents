%------------------------------------------------
\section{Motivations}
%------------------------------------------------

\begin{frame}
\frametitle{Du disque Euclident au disque discret}
\only<1>
{
  \begin{columns}[t]
    \begin{column}{0.5\linewidth}
      \begin{block}{Disque Euclidien : $\mathcal{D}_e$}
        $$\left\{ (x,y) \in \mathbb{R}^{2} | (x - u)^2 + (y - v)^2 \leq R^2 \right\}$$
        \begin{figure}[h!]
          \centering
          \includegraphics[width=0.6\linewidth]{fig/2-mot/circle/circle-euc-0.pdf}
        \end{figure}
      \end{block}
    
    \end{column}
    \begin{column}{0.5\linewidth}
      \begin{block}{Disque Discret : $\mathcal{D}$}
        $$\left\{ (x,y) \in \alert{\mathbb{Z}^{2}} | (x - u)^2 + (y - v)^2 \leq R^2 \right\}$$

        \begin{figure}[h!]
          \centering
          \includegraphics[width=0.6\linewidth]{fig/2-mot/circle/circle-dis-0.pdf}
        \end{figure}
      \end{block}  
    \end{column}
  \end{columns} 

  \begin{exampleblock}{}
  avec $O(u,v) \in \mathbb{Q}^{2}$ les coordonnées du centre et $R^2 \in \mathbb{Q}$ le rayon.\\
  \end{exampleblock}
}
        
\only<2>
{
  \begin{columns}[t]
    \begin{column}{0.5\linewidth}
      \begin{block}{Cercle Euclidien : $\mathcal{C}_e$}
        $$\left\{ (x,y) \in \mathbb{R}^{2} | (x - u)^2 + (y - v)^2 \alert{=} R^2 \right\}$$
        \begin{figure}[h!]
          \centering
          \includegraphics[width=0.6\linewidth]{fig/2-mot/circle/circle-euc-0.pdf}
        \end{figure}
      \end{block}
    
    \end{column}
    \begin{column}{0.5\linewidth}
      \begin{block}{Cercle Discret : $\mathcal{C}$}
        $$\left\{ (x,y) \in \alert{\mathbb{Z}^{2}} | (x - u)^2 + (y - v)^2 \alert{\stackrel{?}{=}} R^2 \right\}$$
        \vspace{-0.24cm}
        \begin{figure}[h!]
          \centering
          \includegraphics[width=0.6\linewidth]{fig/2-mot/circle/circle-dis-0.pdf}
        \end{figure}
      \end{block}  
    \end{column}
  \end{columns} 

  \begin{exampleblock}{}
  avec $O(u,v) \in \mathbb{Q}^{2}$ les coordonnées du centre et $R^2 \in \mathbb{Q}$ le rayon.\\
  \end{exampleblock}
}

\only<3>
{
  \begin{columns}[t]
    \begin{column}{0.5\linewidth}
      \begin{block}{Cercle Euclidien : $\mathcal{C}_e$}
        $$\left\{ (x,y) \in \mathbb{R}^{2} | (x - u)^2 + (y - v)^2 \alert{=} R^2 \right\}$$
        \begin{figure}[h!]
          \centering
          \includegraphics[width=0.6\linewidth]{fig/2-mot/circle/circle-euc-0.pdf}
        \end{figure}
      \end{block}
    
    \end{column}
    \begin{column}{0.5\linewidth}
      \begin{block}{Cercle Discret : $\mathcal{C}$}
        Nécéssité de passer par la notion de \textbf{voisinage}.
        \vspace{0.24cm}
        \begin{figure}[h!]
          \centering
          \includegraphics[width=0.6\linewidth]{fig/2-mot/circle/circle-dis-0.pdf}
        \end{figure}
      \end{block}  
    \end{column}
  \end{columns} 

  \begin{exampleblock}{}
  avec $O(u,v) \in \mathbb{Q}^{2}$ les coordonnées du centre et $R^2 \in \mathbb{Q}$ le rayon.\\
  \end{exampleblock}
}
  
\end{frame}



\begin{frame}
\frametitle{Le voisinage et la connexité}

\begin{block}{}
  \begin{columns}[t]
    \begin{column}{0.5\linewidth}
      4-voisinage d'un point (u,v) : $\mathcal{V}_4(u,v)$
      $$ \left\{ (x,y) \in \mathbb{Z}^{2} |  |x-u|+|y-u| = 1 \right\}$$
    \end{column}
    \hspace{-1cm}
    \begin{column}{0.5\linewidth}
      8-voisinage d'un point (u,v) : $\mathcal{V}_8(u,v)$
      $$ \left\{ (x,y) \in \mathbb{Z}^{2} |  max(|x-u|,|y-u|) = 1 \right\}$$  
    \end{column}
  \end{columns} 

  \begin{figure}[H]
    \centering
    \includegraphics[width=.3\linewidth]{fig/2-mot/connexe/connexite.pdf}
  \end{figure}
\end{block}

\begin{block}{Cercle Discret 8-connexes et 4-connexes}
  \begin{columns}[t]
    \begin{column}{0.5\linewidth}
      \begin{figure}[H]
        \centering
        \includegraphics[width=.5\linewidth]{fig/2-mot/circle/circle-dis-1a.pdf}

      \end{figure}
      
    \end{column}
    \begin{column}{0.5\linewidth}
      \begin{figure}[H]
        \centering
        \includegraphics[width=.5\linewidth]{fig/2-mot/circle/circle-dis-1b.pdf}
      \end{figure}
    \end{column}
  \end{columns} 

  $$ \mathcal{C}_{*} =  \left\{ (x,y) \in \mathcal{D} | \left( \mathcal{V}_{*}(x,y) \cap \mathcal{D} \right) \neq \mathcal{V}_{*}(x,y) \right\}$$
\end{block}


\end{frame}

\begin{frame}
\frametitle{Énoncé de la problématique}
\begin{block}{Le disque discret est l’union de deux ensembles disjoints}
  \begin{columns}[t]
    \begin{column}{0.65\linewidth}
      \begin{itemize}      
        \item Son intérieur strict :
        $\stackrel{\ \circ}{\mathcal{D}}_{*} =  \left\{ (x,y) \in \mathcal{D} | \mathcal{V}_{*}(x,y) \cap \mathcal{D} = \mathcal{V}_{*}(x,y) \right\}$\\
        Les points possèdent tous leurs quatre plus proches voisins à l’intérieur du disque.
        \item Son bord : \\ 
        $\mathcal{C}_{*} =  \left\{ (x,y) \in \mathcal{D} | \left( \mathcal{V}_{*}(x,y) \cap \mathcal{D} \right) \neq \mathcal{V}_{*}(x,y) \right\}$\\
        Organisation spatiale des points moins triviale.
      \end{itemize}
    \end{column}
 
    \begin{column}{0.35\linewidth}
      \begin{figure}[H]
        \centering
        \includegraphics[width=.7\linewidth]{fig/2-mot/circle/circle-dis-2.pdf}
       \end{figure}
    \end{column}
  \end{columns}
\end{block}


\only<1>
{
  \begin{exampleblock}{Remarques}
    \begin{itemize}
      \item La structure de l’intérieur d'un disque discret est évidente.
      \item Seule l’étude du bord, nous intéresse pour comprendre l’organisation des points des disques discrets.
    \end{itemize}
  \end{exampleblock} 
}
\only<2>
{
  \begin{alertblock}{Problématique}
    Comprendre comment sont organisés spatialement les points du bord d’un disque discret et comment cette structure est déterminée par les paramètres du disque (position et taille) par rapport à la grille sous-jacente.
  \end{alertblock}

}





\end{frame}

\begin{frame}
\frametitle{Les $\alpha$-shapes}

\end{frame}

\begin{frame}
\frametitle{La triangulation de Delaunay}

\end{frame}
