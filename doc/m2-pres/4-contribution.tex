%------------------------------------------------
\section{Contributions}
%------------------------------------------------

\begin{frame}
\frametitle{Relations aux triangulations de Delaunay}


\only<1>
{
  \begin{block}{}
    \textbf{Arêtes d'un disque discret :}\\
	  \begin{itemize}
      \item Un disque discret est convexe.
      \item La séquence des points entre deux sommets de l'enveloppe convexe est un motif ou une répétition de motifs de droite discrète de même pente.
    \end{itemize}
    
	  \textbf{Motifs de droites discrètes :}\\
	  \begin{itemize}
      \item Ce sont des discrétisations de segments de droite dont les extrémités sont des points discrets.
    \end{itemize}
  \end{block}

  \begin{columns}[t]
    \begin{column}{0.5\linewidth}
    	\begin{figure}[H]
	      \centering
	      \includegraphics[width=0.4\linewidth]{fig/4-con/tri/con-motif-1.pdf}
	      \includegraphics[width=0.25\linewidth]{fig/4-con/tri/con-disque.pdf}	      
	    \end{figure}

    \end{column}
    \begin{column}{0.5\linewidth}
      \begin{thebibliography}{alpha}
      \scriptsize
        \bibitem{roussillonPR2011}
        [RS11] T. Roussillon and I. Sivignon
        \newblock Faithful Polygonal Representation of the Convex and Concave Parts of a Digital Curve
        \newblock {\em Pattern Recognition}, 2011.    
        \bibitem{RoussillonL11}
        [RL11] Tristan Roussillon and Jacques-Olivier Lachaud
        \newblock Delaunay Properties of Digital Straight Segments
        \newblock {\em Springer : Discrete Geometry for Computer Imagery - 16th {IAPR}}, 2011.
      \end{thebibliography}
      \scriptsize     
    \end{column}
  \end{columns}   
}

\only<2,3>
{
  \begin{block}{}
	  \begin{enumerate}
		\item Les arêtes de l'enveloppe convexe se décomposent en motifs.
		\item La triangulation de Delaunay d'un motif est connue.
	  \item<3> Elle est liée au convergent.
	  \end{enumerate}
  \end{block}

}
\only<2>
{
	\begin{figure}[H]
	  \centering
	  \includegraphics[width=0.5\linewidth]{fig/4-con/tri/con-motif-0.pdf}
	  \includegraphics[width=0.5\linewidth]{fig/4-con/tri/con-conv-0.pdf}
	\end{figure}
}
\only<3>
{
	\begin{figure}[H]
	  \centering
	  \includegraphics[width=0.5\linewidth]{fig/4-con/tri/con-motif-10.pdf}
	  \includegraphics[width=0.5\linewidth]{fig/4-con/tri/con-conv-10.pdf}
	\end{figure}
}

\end{frame}


%-----------------------------------------------------------------
\subsection{$\alpha$-shape, $\alpha \leq 0$ - Généralisation de Har-Peled}
%-----------------------------------------------------------------

\begin{frame}
\frametitle{$\alpha$-shape, $\alpha \leq 0$ - Généralisation de Har-Peled - Présentation}

\begin{block}{Principe}
  La méthode de calcul de l'$\alpha$-shape pour $\alpha <0$ reproduit le schéma du calcul de l’enveloppe convexe.\\ 
  Une étape supplémentaire pour chaque convergent à l'intérieur du disque afin de contrôler la possibilité d'avoir obtenu un sommet de l'$\alpha$-shape.

\end{block}

\end{frame}

\begin{frame}
\frametitle{Algorithme - départ}

\begin{columns}[t]
  \begin{column}{0.7\linewidth}
    \only<1>
    {
      \begin{block}{Point de départ : a}
        Le point d'abscisse minimal et d'ordonnée maximal appartient à l'enveloppe convexe donc il appartient à l'$\alpha$-shape.\\
      \end{block}
    }
    \only<1,2>
    {  
      \begin{block}{Calcul des convergents}
        $p_{-2} = (1,0)$ et $p_{-1} = (0,1)$.\\
        $p_{k} = p_{k-2} + q_k p_{k-1}$\\
        $q_k$ est le plus grand entier tel que $p_{k}$ et $p_{k-2}$ soient dans le même domaine.
      \end{block}
    }
    \only<2,3>
    {
      \begin{block}{Convergent à l'intérieur du disque}
        \begin{itemize}
          \item k impaire.
          \item k paire et $p_k$ exactement sur le bord du disque.
        \end{itemize}
        Vérification de si $p_k$ est un sommet de l'$\alpha$-shape.
        On utilise un \textbf{prédicat}.
      \end{block}
    }
    \only<3,4>
    {
      \begin{block}{Prédicat}
      \begin{itemize}
        \item \textbf{$R_T$} : Rayon du cercle circonscrit au triangle : $T(a, b, c)$.
        \item \textbf{$R_{\alpha}$} $= -1/\alpha$ : Rayon de notre disque généralisé.
      \end{itemize}      
        Compare la longueur de \textbf{$R_T$} par rapport à \textbf{$R_{\alpha}$}.\\
      \end{block}
      
      \begin{alertblock}{Deux cas}
        \begin{itemize}
          \item $\alpha = \alpha_{2}$ (en orange), \textbf{$|R_{\alpha_{2}}| > R_T$}
          \item $\alpha = \alpha_{1}$ (en violet), \textbf{$|R_{\alpha_{1}}| < R_T$}
        \end{itemize}
       \end{alertblock}
    }

  \end{column}
  \begin{column}{0.3\linewidth}
    \only<1>
    {
      \begin{figure}[H]
        \centering
        \includegraphics[width=.9\linewidth]{fig/4-con/nas/con-nas-0.pdf}
      \end{figure}
    }
    \only<2>
    {
      \begin{figure}[H]
        \centering
        \includegraphics[width=.9\linewidth]{fig/4-con/nas/con-nas-1.pdf}
      \end{figure}
    }
    \only<3>
    {
      \begin{figure}[H]
        \centering
        \includegraphics[width=.9\linewidth]{fig/4-con/nas/con-nas-2.pdf}
      \end{figure}
    }
  \end{column}
\end{columns}
\end{frame}

\begin{frame}
\frametitle{Algorithme - Cas 1 - $\alpha = \alpha_{2}$, $|R_{\alpha_{2}}| > R_T$}
\only<1>
{
  \begin{figure}[H]
    \centering
    \includegraphics[width=.5\linewidth]{fig/4-con/nas/con-nas-2.pdf}
  \end{figure}
}

\begin{block}{}
\begin{itemize}
  \item $b \in \mathcal{D}_{\alpha_2}$
  \item $\exists \mathcal{D}_{\alpha_2}$ avec a et c comme points $\alpha$-adjacents contenant tous les points de $\mathcal{D}$.  
  \item \alert{b n'est pas un sommet de l'$\alpha$-shape.}
\end{itemize}

\end{block}

\only<2>
{
  \begin{block}{}
    On poursuit le calcul des prochains convergents.
  \end{block}

  \begin{figure}[H]
    \centering
    \includegraphics[width=0.25\linewidth]{fig/4-con/nas/con-nas-10.pdf}
    \includegraphics[width=0.25\linewidth]{fig/4-con/nas/con-nas-11.pdf}
  \end{figure}
}

\end{frame}

\begin{frame}
\frametitle{Algorithme - Cas 2 - $\alpha = \alpha_{1}$, $|R_{\alpha_{1}}| < R_T$}
\only<1>
{
  \begin{figure}[H]
    \centering
    \includegraphics[width=.5\linewidth]{fig/4-con/nas/con-nas-2.pdf}
  \end{figure}

  \begin{block}{}
    \begin{itemize}
      \item $b \not\in \mathcal{D}_{\alpha_1}$
      \item $\not\exists \mathcal{D}_{\alpha_1}$, avec a et c comme points $\alpha-extrêmes$ contenant b
      \item a et c ne sont pas des points $\alpha$-adjacents.
      \item \alert{c n'en est pas un sommet de l'$\alpha$-shape.}
      \item \alert{Chercher l'ensemble des sommets entre a et c.}
        Étude des points du segment $[b,c]$.
    \end{itemize}
  \end{block}
}
\only<2>
{
  \begin{block}{Étude des sommets entre a et c.}
    \begin{itemize}
      \item Point : $b_i = p_{k-2} + i*p_{k-1}$
      \item Triangle : $T_i = T(a, b_{i}, b_{i+1})$.
      \item Rayon du cercle circonscrit $R_{T_{i}}$
     \end{itemize}
   
     \begin{enumerate}
       \item \alert{Les rayons $R_{T_{i}}$ sont ordonnés et croissants.}
       \item Recherche dichotomique de i tq : $R_{T_{i}} > R_{\alpha} > R_{T_{i-1}}$
     \end{enumerate}
  \end{block}

  \begin{figure}[H]
    \centering
    \includegraphics[width=.8\linewidth]{fig/4-con/nas/con-nas-20.pdf}
  \end{figure}
}
\only<3>
{
  \begin{block}{$R_{T_{2}} > R_{\alpha} > R_{T_{1}}$}
    \alert{L'ensemble des points $\left\{ b_{i}, \ldots, b_{q_k}, c \right\}$ appartiennent à l'$\alpha$-shape.}
  \end{block}

  \begin{figure}[H]
    \centering
    \includegraphics[width=.8\linewidth]{fig/4-con/nas/con-nas-21.pdf}
  \end{figure}
}
\end{frame}

\begin{frame}
\frametitle{Résultats}

\begin{columns}[t]
  \begin{column}{0.35\linewidth}
    \vspace{-0.8cm}
    \begin{tiny}
    \begin{table}[H]
      \begin{tabular}{|p{0.018cm}|p{0.8cm}|p{0.8cm}|p{0.45cm}|}
        \hline
        R & $-\alpha$   & \#       & $/R^{2/3}$\\
        \hline
        5  & 3,2        & 179,02   & 17,7610\\
        6  & 6,4        & 272,92   & 17,0575\\
        7  & 12,8       & 472,19   & 18,5913\\
        8  & 25,6       & 774,45   & 19,2088\\
        9  & 51,2       & 1,30E+03 & 20,3259\\
        10 & 102,4      & 2,14E+03 & 21,0878\\
        11 & 204,8      & 3,54E+03 & 21,9549\\
        12 & 409,6      & 5,68E+03 & 22,1878\\
        13 & 819,2      & 9,25E+03 & 22,7644\\
        14 & 1638,4     & 1,49E+04 & 23,0413\\
        15 & 3276,8     & 2,38E+04 & 23,2816\\
        16 & 6553,6     & 3,84E+04 & 23,6175\\
        17 & 13107,2    & 6,17E+04 & 23,9124\\
        18 & 26214,4    & 9,89E+04 & 24,1500\\
        19 & 52428,8    & 1,59E+05 & 24,4137\\
        20 & 104857,6   & 2,54E+05 & 24,5914\\
        21 & 209715,2   & 4,06E+05 & 24,7603\\
        22 & 419430,4   & 6,49E+05 & 24,9402\\
        23 & 838860,8   & 1,04E+06 & 25,0730\\
        24 & 1677721,6  & 1,65E+06 & 25,2002\\
        25 & 3355443,2  & 2,63E+06 & 25,3061\\
        26 & 6710886,4  & 4.20E+06 & 25.4220\\
        27 & 13421772,8 & 6.70E+06 & 25.5612\\
        28 & 26843545,6 & 1.07E+07 & 25.7011\\
        \hline
      \end{tabular} 
    \end{table}
    \end{tiny}
  \end{column}
  \begin{column}{0.9\linewidth}
    \vspace{-0.6cm}
    \begin{figure}[H]
      \centering
      \includegraphics[width=.75\linewidth]{fig/4-con/nas/nas.png}
    \end{figure}

  \end{column}
\end{columns}

\only<1>
{
  \begin{exampleblock}{}
    \textbf{Nombre de sommet des $\alpha-shapes$}\\
    La division du nombre de sommets par $R^{2/3}$ croît légèrement.\\
    Les résultats sont conformes à la publication.\\
  \end{exampleblock} 
} 
\only<2>
{
  \begin{alertblock}{}
    \begin{enumerate}
      \item Nous avons développé un algorithme incrémental, ``output sensitive'' pour le calcul des $\alpha$-shapes dans le cas $\alpha \leq 0$. 
      \item Nous souhaitons compléter notre programme afin de pouvoir calculer des $\alpha$-shapes dans le cas $\alpha \geq 0$.
    \end{enumerate}
  \end{alertblock} 
} 
 
\end{frame}

%-----------------------------------------------------------------
\subsection{$\alpha$-shape, $\alpha \leq 0$}
%-----------------------------------------------------------------

\begin{frame}
\frametitle{$\alpha$-shape, $\alpha \leq 0$ - Présentation}
  \begin{block}{Principe}
    \begin{itemize}
      \item Approche ``Bottom-Up'' : Méthode de calcul de l'$\alpha$-shape pour $\alpha > 0$ différente des méthodes précédentes.
      \item L'ensemble des sommets de l'$\alpha$-shape est un sous-ensemble de l'enveloppe convexe.
      \item Sélectionner parmi les sommets de l'enveloppe convexe, ceux qui sont pertinents et enlever les autres.
    \end{itemize}  
  \end{block}
\end{frame}

\begin{frame}
\frametitle{Algorithme - départ}

\only<1>
{
  \begin{block}{}
    \begin{enumerate}
      \item Point de départ :
      \begin{itemize}
        \item Tous les sommets de l'enveloppe convexe n'appartiennent pas obligatoirement à l'alpha-shape.
        \item Nous devons choisir un sommet appartenant au plus petit cercle englobant.
        \item Choix par construction parmi l'un des trois sommets du triangle composant le cercle circonscrit.
      \end{itemize}

      \item  $\mathcal{S}$ = l'ensemble des sommets de l'enveloppe convexe du disque discret.
    \end{enumerate}
  \end{block}
}  
\only<1>
{
  \begin{figure}[H]
    \centering
    \includegraphics[width=.4\linewidth]{fig/4-con/pas/con-pas-0.pdf}
  \end{figure}
}  

\only<2>
{
  \begin{block}{}
    \begin{itemize}
      \item $a,b$ et $c \in \mathcal{S}$, trois sommets successifs.
    \end{itemize} 
    Utilisation d'un \textbf{prédicat} pour savoir si b est un sommet de l'$\alpha$-shape.
  \end{block}
}

\only<2,3>
{
  \begin{figure}[H]
    \centering
    \includegraphics[width=.4\linewidth]{fig/4-con/nas/con-nas-2.pdf}
  \end{figure}
}  

\only<2,3>
{
  \begin{block}{Prédicat}
    \begin{itemize}
      \item \textbf{$R_T$} : Rayon du cercle circonscrit au triangle : $T(a, b, c)$.
      \item \textbf{$R_{\alpha}$} $= -1/\alpha$ : Rayon de notre disque généralisé.
    \end{itemize}      
    Compare la longueur de \textbf{$R_T$} par rapport à \textbf{$R_{\alpha}$}.\\
  \end{block}
}
\only<3>
{
  \begin{alertblock}{Deux cas}
    \begin{itemize}
      \item $\alpha = \alpha_{1}$ (en violet), \textbf{$R_{\alpha_{1}} < R_T$}
      \item $\alpha = \alpha_{2}$ (en orange), \textbf{$R_{\alpha_{2}} > R_T$}
    \end{itemize}
  \end{alertblock}
}

\end{frame}

\begin{frame}
\frametitle{Cas 1 - $\alpha = \alpha_{1}$ - $R_{\alpha_{1}} < R_T$}
\only<1>
{
  \begin{figure}[H]
    \centering
    \includegraphics[width=.5\linewidth]{fig/4-con/nas/con-nas-2.pdf}
  \end{figure}
}

\begin{block}{}
  \begin{itemize}
    \item $b \in \mathcal{D}_{\alpha_1}$
    \item $\exists \mathcal{D}_{\alpha_1}$ avec a et c comme points $\alpha$-adjacents contenant tous les points de $\mathcal{D}$.
    \item \alert{b n'est pas un sommet de l'$\alpha$-shape.}
  \end{itemize}
\end{block}

\only<2>
{
  \begin{block}{}
    \begin{enumerate}
      \item Suppression de b la liste des sommets potentiels
      \item a reste inchangé, b devient c et c devient le sommet suivant de $\mathcal{S}$
      \item Poursuite de la procédure.
    \end{enumerate}
  \end{block}
 
  \begin{figure}[H]
    \centering
    \includegraphics[width=0.4\linewidth]{fig/4-con/pas/con-pas-10.pdf}
    \includegraphics[width=0.4\linewidth]{fig/4-con/pas/con-pas-11.pdf}
  \end{figure}
}
  

\end{frame}

\begin{frame}
\frametitle{Cas 2 - $\alpha = \alpha_{2}$ - $R_{\alpha_{2}} > R_T$}
\only<1>
{
  \begin{figure}[H]
    \centering
    \includegraphics[width=.5\linewidth]{fig/4-con/nas/con-nas-2.pdf}
  \end{figure}
}

\begin{block}{}
  \begin{itemize}
    \item $b \not\in \mathcal{D}_{\alpha_2}$
    \item $\not\exists \mathcal{D}_{\alpha_2}$ avec a et c comme points $\alpha$-adjacents contenant tous les points de $\mathcal{D}$.
    \item a et c ne sont pas des points $\alpha$-adjacents.
    \item a et b sont $\alpha$-adjacents.
    \item \alert{b est un sommet de l'$\alpha$-shape.}
  \end{itemize}
\end{block}

\only<2>
{
  \begin{block}{}
    \begin{enumerate}
      \item b devient a, c devient b et c devient le sommet suivant de $\mathcal{S}$.\\
      \item Poursuite de la procédure.
    \end{enumerate}
  \end{block}
 
  \begin{figure}[H]
    \centering
    \includegraphics[width=0.4\linewidth]{fig/4-con/pas/con-pas-20.pdf}
    \includegraphics[width=0.4\linewidth]{fig/4-con/pas/con-pas-21.pdf}
  \end{figure}
}
\end{frame}

\begin{frame}
\frametitle{Complexité et Perspective}

\begin{block}{ Complexité}
  Le calcul se base sur l'enveloppe convexe selon la méthode de Har-Peled. 
  \begin{itemize}
    \item Parcours d'au maximum deux fois l'ensemble des sommets de l'enveloppe convexe ($O(R^{2/3})$). 
  \end{itemize}
  
  \begin{enumerate}
    \item Calcul linéaire par rapport au nombre de sommet de l'enveloppe convexe avec une complexité en $O(R^{2/3} \log R )$.
    \item La complexité dépend du nombre de sommets de l'enveloppe convexe et non du nombre de sommet de l'$\alpha$-shape.\\
  \end{enumerate}
  \alert{Méthode non ``output sensitive''.} 
\end{block}

\begin{block}{Perspective}
  Adopter une approche ``top-down''. 
  Partir des points du plus petit cercle englobant et ajouter successivement les sommets de l'alpha-shape manquant. 
\end{block}
\end{frame}


