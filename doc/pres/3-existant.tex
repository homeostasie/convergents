%-----------------------------------------------------------------
\section{Existant}
%-----------------------------------------------------------------

%-----------------------------------------------------------------
\subsection{Suivi de bord}
%-----------------------------------------------------------------
\begin{frame}
  \frametitle{Suivi de bord}
      \only<1>
    { 
      \begin{block}{Deux cas}
        \begin{itemize}
          \item 4-Connexes.
          \item 8-connexes.
        \end{itemize}
      \end{block} 
    } 
    \only<2,3>
    {
      \begin{block}{Suivi de bord d'un disque discret.}
        \begin{itemize}
          \item On étudie la position des voisins par rapport au disque.
          \item<3> On en déduit la direction à suivre en fonction du sens de rotation.
        \end{itemize}
      \end{block} 
    }
    
    %pics
    \only<1>
    {   
      \begin{figure}[h!]
        \centering
        \includegraphics[width=0.8\linewidth]{fig/connexe/ex-connexe-0.pdf}
      \end{figure}    
    }
    \only<2>
    {     
      \begin{figure}[h!]
        \centering
        \includegraphics[width=0.8\linewidth]{fig/connexe/ex-connexe-1.pdf}
      \end{figure}    
    }
    \only<3>
    {     
      \begin{figure}[h!]
        \centering
        \includegraphics[width=0.8\linewidth]{fig/connexe/ex-connexe-2.pdf}
      \end{figure}    
    }
\end{frame}

%-----------------------------------------------------------------
\subsection{Enveloppe Convexe}
%-----------------------------------------------------------------

\begin{frame}
  \frametitle{Enveloppe convexe}

  \begin{block}{}
    On obtient une liste de sommets adjacents.\\
    On recherche le meilleurs candidat pour construire la prochaine arête.
  \end{block} 

  \begin{columns}[t]
 	 \begin{column}{3cm}
			\begin{figure}[h!]
	      \centering
	      \includegraphics[width=\linewidth]{fig/convexe/ch-1.pdf}
     	\end{figure}    
    \end{column}

    \begin{column}{3cm}
      \begin{figure}[h!]
		      \centering
		      \includegraphics[width=\linewidth]{fig/convexe/ch-2.pdf}
     	\end{figure}    
    \end{column}
			
		\begin{column}{3cm}
      \begin{figure}[h!]
		      \centering
		      \includegraphics[width=\linewidth]{fig/convexe/ch-3.pdf}
     	\end{figure}    
    \end{column}
        
	\end{columns}  

\end{frame}

%-----------------------------------------------------------------

\begin{frame}
  \frametitle{Har-Peled - Convergents}

	\begin{block}{}
	\begin{tiny}
    [H98] Har-Peled, S.\\
    An output sensitive algorithm for discrete convex hulls\\
    {\em Computational Geometry}, 10(2):125--138, 1998.\\
    \end{tiny}
	\end{block} 


	\begin{columns}[t]
 		\begin{column}{5cm}
  		
			\begin{block}{}
	      Calcul des convergents :
	      \begin{itemize}
	        \item $a = (0,0), b = (y, x)$
	        \item $p_{-2} = (1,0), p_{-1} = (0,1)$
					\item<2-> \alert{$ p_{k} = p_{k-2} + q_{k}*p_{k-1}$}\\
		      avec le plus grand $q_{k} \in \mathbb{Z}$ tq $p_{k}$ et $p_{k-2}$ soient du même côté.
	      \end{itemize}
	    \end{block}
		\end{column}

		\begin{column}{5cm}
			\only<1>
			{
				\begin{figure}[h!]
					\centering
				  \includegraphics[width=0.6\linewidth]{fig/har/har2-1.pdf}
			 	\end{figure}    
			}
			\only<2>
			{
				\begin{figure}[h!]
					\centering
				  \includegraphics[width=0.6\linewidth]{fig/har/har2-2.pdf}
			 	\end{figure}    
			}
			\only<3>
			{
				\begin{figure}[h!]
					\centering
				  \includegraphics[width=0.6\linewidth]{fig/har/har2-3.pdf}
			 	\end{figure}    
			}
	  \end{column}
	\end{columns}  

\end{frame}

%-----------------------------------------------------------------

\begin{frame}
  \frametitle{Har-Peled - Passage à l'enveloppe convexe du cercle}
  		
			\begin{block}{Calcul des convergents}
					On initialise avec les deux premiers convergents : $p_{-2}$ et $p_{1}$.
				\begin{itemize}
	        \item Tous les convergents impairs sont à l'intérieur du disque.
					\item Tous les convergents pairs sont à l'extérieur du disque.
	      \end{itemize}
		
	    \end{block}

				\begin{figure}[h!]
					\centering
				  \includegraphics[width=0.2\linewidth]{fig/cercle/cer-1.pdf}
				  \includegraphics[width=0.2\linewidth]{fig/cercle/cer-2.pdf}
					\includegraphics[width=0.2\linewidth]{fig/cercle/cer-3.pdf}
				  \includegraphics[width=0.2\linewidth]{fig/cercle/cer-4.pdf}
			 	\end{figure}    
      \only<2>
      {
			\begin{block}{}
					\alert{On repart du plus grand convergent impair.}
	    \end{block}
      }

\end{frame}

%-----------------------------------------------------------------

\begin{frame}
  \frametitle{Har-Peled - Résultat}
	\begin{columns}[t]
 		\begin{column}{3.8cm}
  		\begin{block}{Sommets/$r^{2/3}$}
				\begin{tiny}
					\begin{tabular}{|l|c|r|}
						\hline
						Rayons & CH-E0 & CH-E1 \\
						\hline
						32 & 3.50816 & 10.1246 \\
						64 & 3.48625 & 10.635 \\
						128 & 3.45612 & 11.1696 \\
						256 & 3.46525 & 11.5349 \\
						512 & 3.46984 & 11.8908 \\
						1024 & 3.46203 & 12.2068 \\
						2048 & 3.46115 & 12.4457 \\
						4096 & 3.45258 & 12.6752 \\
						8192 & 3.45479 & 12.9217 \\
						16384 & 3.45495 & 13.0297 \\
						32768 & 3.45479 & 13.1866 \\
						65536 & 3.45564 & 13.2766 \\
						131072 & 3.45433 & 13.4516 \\
						262144 & 3.45375 & 13.5341 \\
						524288 & 3.45423 & 13.6428 \\
						1048576 & 3.45374 & 13.7542 \\
						2097152 & 3.45316 & 13.8098 \\
						4194304 & 3.45429 & 13.8796 \\
						8388608 & 3.45336 & 13.9427 \\
						16777216 & 3.45343 & 14.0154 \\
						33554432 & 3.45357 & 14.0745 \\
						67108864 & 3.45092 & 14.0992 \\
						134217728 & 3.42459 & 14.2689 \\
						268435456 & 3.2448 & 15.9038 \\
					  \hline
					\end{tabular}  
				\end{tiny}
			\end{block} 
		\end{column}
		\begin{column}{7cm}
		    \only<1>
		    {
				  \begin{figure}[h!]
					  \centering
				    \includegraphics[width=\linewidth]{fig/res/ch-e1-100vertices-byradius.png}
			   	\end{figure}    
        }
        \only<2>
		    {
				  \begin{figure}[h!]
					  \centering
				    \includegraphics[width=\linewidth]{fig/res/time.png}
			   	\end{figure}    
        }
		\end{column}
	\end{columns} 

\end{frame}


%-----------------------------------------------------------------
\subsection{Triangulation de Delaunay}
%-----------------------------------------------------------------

\begin{frame}
  \frametitle{Triangulation de Delaunay}

	\begin{columns}[t]
 		\begin{column}{7cm}

			\begin{block}{Triangulation des arêtes}
				\begin{itemize}
					\item Étude des motifs des arêtes pour construire leurs triangulations de Delaunay.
					\item Les $\alpha$-shapes sont des sous-graphes de la triangulation. 
	      \end{itemize}	
			\end{block} 

			\only<2>
			{
				\begin{block}{}
					\alert{Construire les alpha-shapes directement en s'appuyant sur les convergents.}
				\end{block}
			}
 		\end{column}
    \begin{column}{3cm}
			\begin{figure}[h!]
				\centering
			  \includegraphics[width=\linewidth]{fig/ex/tri.pdf}
			\end{figure}
	  \end{column}
	\end{columns}
\end{frame}
