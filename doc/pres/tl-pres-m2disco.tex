%----------------------------------------------------------------------------------------
%    PACKAGES AND THEMES
%----------------------------------------------------------------------------------------

\documentclass{beamer}

\usepackage{times}
\usepackage[T1]{fontenc}
\usepackage[english,francais]{babel}
\usefonttheme{professionalfonts}

\usepackage[]{beamerthemeliris} % paquet theme, options: nogradient,nobackground


\usepackage{graphicx} %paquet graphiques
\usepackage{epsfig}

\usepackage{booktabs} % Allows the use of \toprule, \midrule and \bottomrule in tables
\usepackage[utf8]{inputenc} %codage
\usepackage{lmodern}
\usepackage{amssymb}
\usepackage{amsmath}

\usepackage{bibentry}
\nobibliography*

 %----------------------------------------------------------------------------------------
%    TITLE PAGE
%----------------------------------------------------------------------------------------

\title[Présentation M2Disco]{Structure d'un cercle discret en dimension deux} % The short title appears at the bottom of every slide, the full title is only on the title page

\author{Thomas \bsc{Lafond}} % Your name
\date{10 Juin 2013}
\institute[LIRIS] % Your institution as it will appear on the bottom of every slide, may be shorthand to save space
{
  Laboratoire d'InfoRmatique en Image et Syst\`{e}mes d'information \\ % Your institution for the title page
  \medskip
  \textit{-- Présentation d'équipe : M2Disco --}\\
  \medskip
  Encadrant : Tristan \bsc{Roussillon}
}
\date{10 Juin 2013} % Date, can be changed to a custom date

\begin{document}

% titre
\begin{frame}
  \titlepage % Print the title page as the first slide
\end{frame}

%----------------------------------------------------------------------------------------
%    PRESENTATION SLIDES
%----------------------------------------------------------------------------------------

%------------------------------------------------
\section{Parcours}
%------------------------------------------------

\subsection{Parcours}
 
\begin{frame}
%\frametitle{Parcours}
  \begin{block}{Licence de Mathématiques et d'Informatique}
    \begin{itemize}
      \item Mathématiques : L1, L2, L3.
      \item Informatique : L1, L2.
      \item Domaine Scientifique : Science Physique, Chimie et Biologie.
    \end{itemize}
  \end{block}

  \begin{block}{Master Statistiques, Informatique et Techniques Numériques}
    \begin{columns}[t]

      \begin{column}{5.5cm}
        Trois composantes principales :
        \begin{itemize}
          \item Probabilités et Statistiques
          \item EDP et Calcul scientifique
          \item Développement Informatique
        \end{itemize}
      \end{column}

      \begin{column}{4.5cm}
        Une certaine ouverture  
        \begin{itemize}
          \item \alert{Stages}
          \item Calcul parallèle
          \item Mécanique des fluides
        \end{itemize}
      \end{column}
    \end{columns}  
  \end{block}

    \begin{block}{Stages précédents}
      \begin{itemize}
        \item Modélisation Hydrologique - Irstea - 5 Mois
        \item Décomposition de domaine - MmodD / Lyon 1 - 2 Mois
      \end{itemize}
  \end{block}

\end{frame}

% table des matières
\begin{frame}
  \frametitle{Tables des matières} % Table of contents slide, comment this block out to remove it
  \setcounter{tocdepth}{1}
  \tableofcontents %
\end{frame}

%------------------------------------------------
\section{Motivations}
%------------------------------------------------

\subsection{Disque}
\begin{frame}
  %\frametitle{Disque}
  \only<1>
  {
    \begin{block}{Définition du disque Eucliden}
      $$\mathcal{D} =  \left\{ (x,y) \in \mathbb{R}^{2} |  ax + by + c(x^2 + y^2) + d \leq 0. \right\}$$
    \end{block}
  }
  \only<2>
  {
    \begin{block}{Définition du bord du disque Eucliden}
      $$\mathcal{D} =  \left\{ (x,y) \in \mathbb{R}^{2} |  ax + by + c(x^2 + y^2) + d \alert{=} 0. \right\}$$
    \end{block}
  }  
  \only<3->
  {
    \begin{block}{Définition du disque discret}
      $$\mathcal{D} =  \left\{ (x,y) \in \alert{\mathbb{Z}^{2}} |  ax + by + c(x^2 + y^2) + d \leq 0. \right\}$$
    \end{block}
  }
  \begin{columns}[t]
    \begin{column}{6cm}
      \only<1>
      {
        \begin{figure}[h!]
          \centering
          \includegraphics[width=4cm]{fig/mot-cercle-euc.pdf}
        \end{figure}
      }
      \only<2>
      {
        \begin{figure}[h!]
          \centering
          \includegraphics[width=4cm]{fig/mot-cercle-euc-1.pdf}
        \end{figure}
      }
      \only<3->
      {
        \begin{figure}[h!]
          \centering
          \includegraphics[width=4cm]{fig/mot-cercle-dis.pdf}
        \end{figure}
      }
      \end{column}
      \begin{column}{5cm}
        \vspace{-0.4cm}
        \begin{exampleblock}{Placement d'un point :}
          \begin{itemize}
            \item Intérieur
            \item Extérieur
            \item Bord
          \end{itemize}
        \end{exampleblock}

        \only<4>
        {
          \begin{block}{Question :}
            \begin{center}  
              \alert{Comment sont représenter les contours de disques discrets dans $\mathbb{Z}^{2}$ ?}
            \end{center}
          \end{block}
        }
      \end{column}
    \end{columns} 
\end{frame}

\subsection{$\alpha$-shape}
\begin{frame}
%\frametitle{Les $\alpha$-shape}
  \begin{block}{$\alpha$-shape de $\mathcal{D}$}
    \begin{itemize}
      \item $\alpha$-hull : Intersection de tous les disques généralisés de rayon $1/\alpha$ qui contiennent tous les points de $\mathcal{D}$.
      \item $\alpha$-extreme : Point appartenant au bord de l'$\alpha$-hull.
      \item $\alpha$-shape : Enveloppe reliant tous les $\alpha$-extreme voisins entre eux.
    \end{itemize}
  \end{block}
\begin{columns}[t]
  \begin{column}{4cm}
    \begin{exampleblock}{Cercle}
      $\mathcal{C} \left( (24,-8), (16,1), (0,0) \right)$\\
       
      \only<1>
      {
        Centre : (8.86, -13.4)\\
        R : 16.06
      }
      \only<2>
      {
        $\alpha = -1$\\
         Nb Sommets : 126\\
      }
      \only<3>
      {
        $\alpha = -1/\sqrt{2}$\\
        Nb Sommets : 89\\
      }
      \only<4>
      {
        $\alpha = -1/\sqrt{10}$\\
         Nb Sommets : 71\\
      }
      \only<5>
      {
        $\alpha = -1/\sqrt{20}$\\
         Nb Sommets : 64\\
      }
      \only<6>
      {
        $\alpha = -1/\sqrt{80}$\\
         Nb Sommets : 61\\
      }
      \only<7>
      {
        $\alpha = 0$\\
         Nb Sommets : 23\\
      }
      \only<8>
      {
        $\alpha = 1/\sqrt{600}$\\
         Nb Sommets : 21\\
      }
      \only<9>
      {
        $\alpha = 1/\sqrt{400}$\\
         Nb Sommets : 19\\
      }
      \only<10>
      {
        $\alpha = 1/\sqrt{300}$\\
         Nb Sommets : 11\\
      }
      \only<11>
      {
        $\alpha = 1/\sqrt{260}$\\
         Nb Sommets : 6\\
      }
    \end{exampleblock}

    
  \end{column}

  \begin{column}{6cm}
    \vspace{-0.8cm}     
    \only<1>
    {
      \begin{figure}[h!]
        \centering
        \includegraphics[width=5.5cm]{fig/as/as-0.pdf}
      \end{figure}
    }
    \only<2>
    {
      \begin{figure}[h!]
        \centering
        \includegraphics[width=5.5cm]{fig/as/as-1.pdf}
      \end{figure}
    }
    \only<3>
    {
      \begin{figure}[h!]
        \centering
        \includegraphics[width=5.5cm]{fig/as/as-2.pdf}
      \end{figure}
    }
    \only<4>
    {
      \begin{figure}[h!]
        \centering
        \includegraphics[width=5.5cm]{fig/as/as-3.pdf}
      \end{figure}
    }
    \only<5>
    {
      \begin{figure}[h!]
        \centering
        \includegraphics[width=5.5cm]{fig/as/as-4.pdf}
      \end{figure}
    }
    \only<6>
    {
      \begin{figure}[h!]
        \centering
        \includegraphics[width=5.5cm]{fig/as/as-5.pdf}
      \end{figure}
    }
    \only<7>
    {
      \begin{figure}[h!]
        \centering
        \includegraphics[width=5.5cm]{fig/as/as-6.pdf}
      \end{figure}
    }
    \only<8>
    {
      \begin{figure}[h!]
        \centering
        \includegraphics[width=5.5cm]{fig/as/as-7.pdf}
      \end{figure}
    }
    \only<9>
    {
      \begin{figure}[h!]
        \centering
        \includegraphics[width=5.5cm]{fig/as/as-8.pdf}
      \end{figure}
    }
    \only<10>
    {
      \begin{figure}[h!]
        \centering
        \includegraphics[width=5.5cm]{fig/as/as-9.pdf}
      \end{figure}
    }
    \only<11>
    {
      \begin{figure}[h!]
        \centering
        \includegraphics[width=5.5cm]{fig/as/as-10.pdf}
      \end{figure}
    }

  \end{column}
\end{columns} 

\end{frame}

%------------------------------------------------
\section{Existant}
%------------------------------------------------

\subsection{Enveloppe convexe - Har-Peled}
\begin{frame}
\frametitle{Geometric-$GCD(x, y)$ - Har-Peled }
\begin{columns}[t]
  \begin{column}{6cm}
    \begin{block}{}

      Calcul des convergents :
      \begin{itemize}
        \item $a = (0,0), b = (y, x)$
        \item $p_{-2} = (1,0), p_{-1} = (0,1)$
      \end{itemize}

      Calcul récursif des suivants :
      \vspace{-0.2cm} 
      \alert{$$ p_{k} = p_{k-2} + q_{k}*p_{k-1}$$}
      avec le plus grand $q_{k} \in \mathbb{Z}$ tq $p_{k}$ et $p_{k-2}$ soient du même côté.\\
      \vspace{0.2cm}
      | Tant que $ p_{k} \in d$.
    \end{block}
    \begin{alertblock}{}
      \vspace{-0.2cm} 
      $$ PGCD(x, y) = x/ x_{n} = y/ y_{n}$$
      \vspace{-0.2cm} 
    \end{alertblock}
  \end{column}

  \begin{column}{4cm}
    
  \only<1>
  {
    \begin{figure}[h!]
      \centering
      \includegraphics[width=3.5cm]{fig/har/har-1-0.pdf}
    \end{figure}
  }
  \only<2>
  {
    \begin{figure}[h!]
      \centering
      \includegraphics[width=3.5cm]{fig/har/har-1-1.pdf}
    \end{figure}
  }
  \only<3>
  {
    \begin{figure}[h!]
      \centering
      \includegraphics[width=3.5cm]{fig/har/har-1-2.pdf}
    \end{figure}
  }
  \only<4>
  {
    \begin{figure}[h!]
      \centering
      \includegraphics[width=3.5cm]{fig/har/har-1-3.pdf}
    \end{figure}
  }
  \only<5>
  {
    \begin{figure}[h!]
      \centering
      \includegraphics[width=3.5cm]{fig/har/har-1-4.pdf}
    \end{figure}
  }
  \only<6>
  {
    \begin{figure}[h!]
      \centering
      \includegraphics[width=3.5cm]{fig/har/har-1-5.pdf}
    \end{figure}
  }
  \only<7>
  {
    \begin{figure}[h!]
      \centering
      \includegraphics[width=3.5cm]{fig/har/har-1-6.pdf}
    \end{figure}
  }
  \only<8>
  {
    \begin{figure}[h!]
      \centering
      \includegraphics[width=3.5cm]{fig/har/har-1-7.pdf}
    \end{figure}
  }
  \only<9>
  {
    \begin{figure}[h!]
      \centering
      \includegraphics[width=3.5cm]{fig/har/har-1-8.pdf}
    \end{figure}
  }   
  \end{column}
\end{columns} 
\end{frame}

\begin{frame}
\frametitle{Calcul de l'enveloppe convexe d'un cercle}
\begin{columns}[t]
  \begin{column}{4cm}
  
  \begin{block}{Différences}
    \begin{itemize}
      \item $a\ne (0, 0)$.
      \item $b$ inconnu.
    \end{itemize}
  \end{block}
  
    \begin{block}{Algorithme}
    Trouver $a_0$
    Calculer les convergents.\\
    Récolter les candidats :
    \begin{itemize}
      \item Intérieur : $k / 2 \equiv 1 [2]$ 
      \item Sur le bord
    \end{itemize}
    Choisir le meilleurs :\\
    \alert{Déterminant$(\leftarrow{u}, \leftarrow{v})$}\\
    | Tant que $ a_{i} \ne a_0$.
 
  \end{block}
    
  \end{column}

  \begin{column}{6cm}
    \only<1>
    {
      \begin{figure}[h!]
        \centering
        \includegraphics[width=6cm]{fig/cercle/cercle-0.pdf}
        \caption{Départ de $a_{0}$}
    \end{figure}
    }
    \only<2>
    {
      \begin{figure}[h!]
        \centering
        \includegraphics[width=6cm]{fig/cercle/cercle-1.pdf}
        \caption{$p_{-2}, p_{-1}$}
    \end{figure}
    }
    \only<3>
    {
      \begin{figure}[h!]
        \centering
        \includegraphics[width=6cm]{fig/cercle/cercle-2.pdf}
        \caption{$p_{-1}$ candidat}
    \end{figure}
    }
    \only<4>
    {
      \begin{figure}[h!]
        \centering
        \includegraphics[width=6cm]{fig/cercle/cercle-3.pdf}
        \caption{$p_{0} = p_{-2}$}
    \end{figure}
    }
    \only<5>
    {
      \begin{figure}[h!]
        \centering
        \includegraphics[width=6cm]{fig/cercle/cercle-4.pdf}
        \caption{Calcul de $p_{1}$}
    \end{figure}
    }
    \only<6>
    {
      \begin{figure}[h!]
        \centering
        \includegraphics[width=6cm]{fig/cercle/cercle-5.pdf}
        \caption{$p_{1}$ candidat}
    \end{figure}
    }
    \only<7>
    {
      \begin{figure}[h!]
        \centering
        \includegraphics[width=6cm]{fig/cercle/cercle-6.pdf}
        \caption{Calcul de $p_{2}$}
    \end{figure}
    }
    \only<8>
    {
      \begin{figure}[h!]
        \centering
        \includegraphics[width=6cm]{fig/cercle/cercle-7.pdf}
        \caption{$p_{2}$}
    \end{figure}
    }
    \only<9>
    {
      \begin{figure}[h!]
        \centering
        \includegraphics[width=6cm]{fig/cercle/cercle-8.pdf}
        \caption{$p_{2}$ candidat ?}
    \end{figure}
    }
    \only<10>
    {
      \begin{figure}[h!]
        \centering
        \includegraphics[width=6cm]{fig/cercle/cercle-9.pdf}
        \caption{Calcul de $p_{3}$}
    \end{figure}
    }
    \only<11>
    {
      \begin{figure}[h!]
        \centering
        \includegraphics[width=6cm]{fig/cercle/cercle-10.pdf}
        \caption{$p_{3}$ candidat}
    \end{figure}
    }
    \only<12>
    {
      \begin{figure}[h!]
        \centering
        \includegraphics[width=6cm]{fig/cercle/cercle-11.pdf}
        \caption{Calcul de $p_{4}$ ?}
    \end{figure}
    }
    \only<13>
    {
      \begin{figure}[h!]
        \centering
        \includegraphics[width=6cm]{fig/cercle/cercle-12.pdf}
        \caption{$p_{4}$ n'existe pas}
    \end{figure}
    }
    \only<14>
    {
      \begin{figure}[h!]
        \centering
        \includegraphics[width=6cm]{fig/cercle/cercle-13.pdf}
        \caption{Prochain départ}
    \end{figure}
    }
    \only<15>
    {
      \begin{figure}[h!]
        \centering
        \includegraphics[width=6cm]{fig/cercle/cercle-14.pdf}
        \caption{On recommence}
    \end{figure}
    }

  \end{column}
\end{columns} 

\end{frame}
    
\subsection{Triangulation de Delaunay}  
\begin{frame}
%\frametitle{Triangulation de Delaunay - \cite{EdeKirSei83}}
  \begin{block}{Triangulation de Delaunay - \cite{EdeKirSei83}}
    \alert{Closest :} \\
    \hspace{0.5cm}Aucun point de S n'appartient à un cercle circonscrit issue de la triangulation closest de Delaunay.\\
    \alert{Furthest :} \\
    \hspace{0.5cm}Tous les points de S appartiennent à tous les cercles circonscrits issue de la triangulation furthest de Delaunay.
      
  \end{block}

  \begin{exampleblock}{Passage aux $\alpha$-shape}
    Les $\alpha$-shape sont des sous-graphes de la triangulation de Delaunay.
    \textbf{Algorithme de construction :} (CGAL)
    \begin{itemize}
      \item Construire la triangulation ($DT_c$ ou $DT_f$).
      \item Chercher les $\alpha$-extrême.
      \item Construires les polygônes.
    \end{itemize}
  \end{exampleblock}
  


\end{frame}

%------------------------------------------------
\section{Contributions}
%------------------------------------------------

\subsection{Généralisation de Har-Peled - $\alpha \leq 0$}
\begin{frame}
\frametitle{Généralisation de Har-Peled - $\alpha \leq 0$}
  
  \begin{columns}[t]
  \begin{column}{7cm}
      \only<1>
      {
        \begin{figure}[h!]
          \centering
          \includegraphics[trim = 3cm 0cm 10cm 2cm, clip, width=7cm]{fig/as-algo/as-algo-1.pdf}
      \end{figure}
      }
      \only<2>
      {
        \begin{figure}[h!]
          \centering
          \includegraphics[trim = 3cm 0cm 10cm 2cm, clip, width=7cm]{fig/as-algo/as-algo-2.pdf}
      \end{figure}
      }
      \only<3>
      {
        \begin{figure}[h!]
          \centering
          \includegraphics[trim = 3cm 0cm 10cm 2cm, clip, width=7cm]{fig/as-algo/as-algo-3.pdf}
      \end{figure}
      }
      \only<4>
      {
        \begin{figure}[h!]
          \centering
          \includegraphics[trim = 3cm 0cm 10cm 2cm, clip, width=7cm]{fig/as-algo/as-algo-4.pdf}
      \end{figure}
      }
      \only<5>
      {
        \begin{figure}[h!]
          \centering
          \includegraphics[trim = 3cm 0cm 10cm 2cm, clip, width=7cm]{fig/as-algo/as-algo-10.pdf}
      \end{figure}
      }
      \only<6>
      {
        \begin{figure}[h!]
          \centering
          \includegraphics[trim = 3cm 0cm 10cm 2cm, clip, width=7cm]{fig/as-algo/as-algo-11.pdf}
      \end{figure}
      }
      \only<7>
      {
        \begin{figure}[h!]
          \centering
          \includegraphics[trim = 3cm 0cm 10cm 2cm, clip, width=7cm]{fig/as-algo/as-algo-12.pdf}
      \end{figure}
      }
      \only<8>
      {
        \begin{figure}[h!]
          \centering
          \includegraphics[trim = 3cm 0cm 10cm 2cm, clip, width=7cm]{fig/as-algo/as-algo-12a.pdf}
      \end{figure}
      }
      \only<9>
      {
        \begin{figure}[h!]
          \centering
          \includegraphics[trim = 3cm 0cm 10cm 2cm, clip, width=7cm]{fig/as-algo/as-algo-13.pdf}
      \end{figure}
      }
      \only<10>
      {
        \begin{figure}[h!]
          \centering
          \includegraphics[trim = 3cm 0cm 10cm 2cm, clip, width=7cm]{fig/as-algo/as-algo-15.pdf}
      \end{figure}
      }
      \only<11>
      {
        \begin{figure}[h!]
          \centering
          \includegraphics[trim = 3cm 0cm 10cm 2cm, clip, width=7cm]{fig/as-algo/as-algo-20.pdf}
      \end{figure}
      }
      \only<12>
      {
        \begin{figure}[h!]
          \centering
          \includegraphics[trim = 3cm 0cm 10cm 2cm, clip, width=7cm]{fig/as-algo/as-algo-21.pdf}
      \end{figure}
      }
      \only<13>
      {
        \begin{figure}[h!]
          \centering
          \includegraphics[width=7cm]{fig/as-algo/dicho.pdf}
      \end{figure}
      }
      
      \only<14>
      {
        \begin{figure}[h!]
          \centering
          \includegraphics[trim = 3cm 0cm 10cm 2cm, clip, width=7cm]{fig/as-algo/as-algo-22.pdf}
      \end{figure}
      }
      \only<15>
      {
        \begin{figure}[h!]
          \centering
          \includegraphics[trim = 3cm 0cm 10cm 2cm, clip, width=7cm]{fig/as-algo/as-algo-23.pdf}
      \end{figure}
      }
      \only<16>
      {
        \begin{figure}[h!]
          \centering
          \includegraphics[trim = 3cm 0cm 10cm 2cm, clip, width=7cm]{fig/as-algo/as-algo-24.pdf}
      \end{figure}
      }
      \only<17>
      {
        \begin{figure}[h!]
          \centering
          \includegraphics[trim = 3cm 0cm 10cm 2cm, clip, width=7cm]{fig/as-algo/as-algo-25.pdf}
      \end{figure}
      }
      \only<18>
      {
        \begin{figure}[h!]
          \centering
          \includegraphics[trim = 3cm 0cm 10cm 2cm, clip, width=7cm]{fig/as-algo/as-algo-26.pdf}
      \end{figure}
      }
      \only<19>
      {
        \begin{figure}[h!]
          \centering
          \includegraphics[trim = 3cm 0cm 10cm 2cm, clip, width=7cm]{fig/as-algo/as-algo-27.pdf}
      \end{figure}
      }
      \only<20>
      {
        \begin{figure}[h!]
          \centering
          \includegraphics[trim = 3cm 0cm 10cm 2cm, clip, width=7cm]{fig/as-algo/as-algo-28.pdf}
      \end{figure}
      }
      \only<21>
      {
        \begin{figure}[h!]
          \centering
          \includegraphics[trim = 3cm 0cm 10cm 2cm, clip, width=7cm]{fig/as-algo/as-algo-29.pdf}
      \end{figure}
      }
      \only<22>
      {
        \begin{figure}[h!]
          \centering
          \includegraphics[trim = 3cm 0cm 10cm 2cm, clip, width=7cm]{fig/as-algo/as-algo-30.pdf}
      \end{figure}
      }
      \only<23>
      {
        \begin{figure}[h!]
          \centering
          \includegraphics[trim = 3cm 0cm 10cm 2cm, clip, width=7cm]{fig/as-algo/as-algo-31.pdf}
      \end{figure}
      }
      \only<24>
      {
        \begin{figure}[h!]
          \centering
          \includegraphics[trim = 3cm 0cm 10cm 2cm, clip, width=7cm]{fig/as-algo/as-algo-32.pdf}
      \end{figure}
      }
    \end{column}
    \begin{column}{3cm}
      \begin{block}{}
        \only<1>
        {
          On cherche le premier point.\\
          À l'extrémité inférieur droite. 
        }
        \only<2,3>
        {
          $p_{-2} = (1, 0)$\\
          $p_{-1} = (0, 1)$\\
        }
        \only<3,4,5,6>
        {
          $p_{0} = (1, 0)$\\
          $p_{1} = (5, 1)$\\
        }
        \only<4,5>
        {
          Triangle : \\
          $T(a, p_{1}, p_{1}-p_{0})$\\
          On s'intéresse à son rayon.\\
        }
        \only<5>
        {
          On a :\\
          $R_{\alpha} > R_T$
        }
        \only<6>
        {
          On continue la recherche de convergent.\\
          $p_{2}$ n'existe pas.\\
          On reprends à partir de de $p_1$.\\
        }
        \only<7>
        {
          Nouveau point de départ.\\
          $p_{-2} = (1, 0)$\\
          $p_{-1} = (0, 1)$\\
          $p_{0} = (1, 0)$\\
          $p_{1} = (3, 1)$\\
        }
        \only<8>
        {
          On a encore:\\
          $R_{\alpha} > R_T$
        }
        \only<9>
        {
          D'où des nouveaux sommets.\\
        }
        \only<10>
        {
          On retrouve le calcul de l'enveloppe Convexe.\\
        }
        \only<11, 12>
        {
          Supposons maintentant que :
          $R_{\alpha} < R_T$\\
        }
        \only<12>
        {
          L'enveloppe ne peut atteindre $p_{1}$ directement sans passer par au moins un autre point.\\
        }
        \only<13>
        {
          Comme les rayons sont ordonnés et strictement croissant.\\
          On cherche par dichotomie le premier rayon plus grand que le prédicat.\\
        }        
        \only<14>
        {
          Si $R_{\alpha} < R_T$\\
          On va chercher en aval.
        }
        \only<15>
        {
          Si $R_{\alpha} > R_T$\\
          On va chercher en amont.
        }
        \only<16, 17>
        {
          Jusqu'à trouver le premier point dans l'enveloppe...\\          
        }
        \only<17>
        {
          et tous les suivants jusqu'à $p_1$.     
        }
        \only<18>
        {
          $p_1$ qui devient le nouveau point de départ.
        }
        \only<19>
        {
          On recommence la méthode next avec la recherche de convergent.\\
          $p_{-2} = (1, 0)$\\
          $p_{-1} = (0, 1)$\\
          $p_{0} = (1, 0)$\\
          $p_{1} = (3, 1)$\\
        }
        \only<20>
        {
          $R_{\alpha} < R_T$\\
          On lance la méthode dichotomique. 
        }
        \only<21>
        {
          Comme $R_{\alpha} > R_T$\\
          On va chercher en amont...
        }
        \only<22,23>
        {
          On ajoute les points à notre $\alpha$-shape.\\
        }
        \only<24>
        {
          On poursuit la méthode jusqu'à retrouver le premier point.
        }        
      \end{block}     
    \end{column}
  \end{columns} 

 
\end{frame}

\subsection{$\alpha \geq 0$}
\begin{frame}
\frametitle{$\alpha \geq 0$}
\end{frame}

\subsection{Protocole}
\begin{frame}
\frametitle{Protocole}
\end{frame}

%------------------------------------------------
\section{Résultats}
%------------------------------------------------

\subsection{Enveloppe convexe}
\begin{frame}
\frametitle{Enveloppe convexe}
\end{frame}

\subsection{$\alpha \leq 0$}
\begin{frame}
\frametitle{$\alpha \leq 0$}
\end{frame}

\subsection{$\alpha \geq 0$}
\begin{frame}
\frametitle{$\alpha \geq 0$}
\end{frame}

%------------------------------------------------
\section{Suite}
%------------------------------------------------

\subsection{Poursuite du projet}
\begin{frame}
\frametitle{Poursuite du projet}
\end{frame}

\subsection{Poursuite personnelle}
\begin{frame}
\frametitle{Poursuite personnelle}
\bibliographystyle{alpha}
\bibliography{tl_bibliographie.bib}
\end{frame}

%----------------------------------------------------------------------------------------
\end{document} 
