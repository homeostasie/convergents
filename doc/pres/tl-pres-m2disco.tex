%----------------------------------------------------------------------------------------
%    PACKAGES AND THEMES
%----------------------------------------------------------------------------------------

\documentclass{beamer}

\usepackage{times}
\usepackage[T1]{fontenc}
\usepackage[english,francais]{babel}
\usefonttheme{professionalfonts}

\usepackage[]{beamerthemeliris} % paquet theme, options: nogradient,nobackground


\usepackage{graphicx} %paquet graphiques
\usepackage{epsfig}

\usepackage{booktabs} % Allows the use of \toprule, \midrule and \bottomrule in tables
\usepackage[utf8]{inputenc} %codage
\usepackage{lmodern}
\usepackage{amssymb}
\usepackage{amsmath}

\usepackage{bibentry}
\nobibliography*

 %----------------------------------------------------------------------------------------
%    TITLE PAGE
%----------------------------------------------------------------------------------------

\title[Présentation M2Disco]{Structure du bord d'un disque discret} % The short title appears at the bottom of every slide, the full title is only on the title page

\author{Thomas \bsc{Lafond}} % Your name
\date{10 Juin 2013}
\institute[LIRIS] % Your institution as it will appear on the bottom of every slide, may be shorthand to save space
{
  Laboratoire d'InfoRmatique en Image et Syst\`{e}mes d'information \\ % Your institution for the title page
  \medskip
  \textit{-- Présentation d'équipe : M2Disco --}\\
  \medskip
  Encadrant : Tristan \bsc{Roussillon}
}
\date{10 Juin 2013} % Date, can be changed to a custom date

\begin{document}

% titre
\begin{frame}
  \titlepage % Print the title page as the first slide
\end{frame}

%----------------------------------------------------------------------------------------
%    PRESENTATION SLIDES
%----------------------------------------------------------------------------------------

%------------------------------------------------
\section{Parcours}
%------------------------------------------------

\begin{frame}
%\frametitle{Parcours}
   \begin{block}{Licence de Mathématiques et d'Informatique}
    \begin{itemize}
      \item Mathématiques : L1, L2, L3.
      \item Informatique : L1, L2.
      \item Domaine Scientifique : Science Physique, Chimie et Biologie.
    \end{itemize}
  \end{block}

  \begin{block}{Master Statistiques, Informatique et Techniques Numériques}
    \begin{columns}[t]
      \begin{column}{5.5cm}
        Trois composantes principales :
        \begin{itemize}
          \item Probabilités et Statistiques
          \item EDP et Calcul scientifique
          \item Développement Informatique
        \end{itemize}
      \end{column}
      \begin{column}{4.5cm}
        Une certaine ouverture  
        \begin{itemize}
          \item \alert{Stages}
          \item Calcul parallèle
          \item Mécanique des fluides
        \end{itemize}
      \end{column}
    \end{columns}  
  \end{block}

  \begin{block}{Stages précédents}
    \begin{itemize}
      \item Modélisation Hydrologique - Irstea - 5 Mois
      \item Décomposition de domaine - MmodD / Lyon 1 - 2 Mois
    \end{itemize}
  \end{block}
\end{frame}


% table des matières
\AtBeginSection[]{
  \begin{frame}{Sommaire}
  \small \tableofcontents[currentsection, hideallsubsections]
  \end{frame} 
}


%------------------------------------------------
\section{Motivations}
%------------------------------------------------

%-----------------------------------------------------------------
\subsection{Disques discret}
%-----------------------------------------------------------------

\begin{frame}
\frametitle{Construction de la problématique}
  \only<1>
  {
    \begin{block}{Définition du disque Euclidien}
      $$\mathcal{D} =  \left\{ (x,y) \in \mathbb{R}^{2} |  ax + by + c(x^2 + y^2) + d \leq 0. \right\}$$
    \end{block}
    
    \begin{figure}[h!]
      \centering
      \includegraphics[width=4cm]{fig/2-motivation/circle/mot-cercle-euc.pdf}
    \end{figure}
        
  }
  \only<2->
  {
    \begin{block}{Définition du disque discret}
      $$\mathcal{D} =  \left\{ (x,y) \in \alert{\mathbb{Z}^{2}} |  ax + by + c(x^2 + y^2) + d \leq 0. \right\}$$
    \end{block}
  }
  \only<2>
  {  
    \begin{figure}[h!]
      \centering
      \includegraphics[width=4cm]{fig/2-motivation/circle/mot-cercle-dis.pdf}
    \end{figure}
  }
  
  \begin{columns}[t]
    \begin{column}{6cm}
      \only<3>
      {
        \begin{figure}[h!]
          \centering
          \includegraphics[width=4cm]{fig/2-motivation/circle/mot-cercle-dis.pdf}
        \end{figure}
      }
      \only<4>
      {
        \begin{figure}[h!]
          \centering
          \includegraphics[width=4cm]{fig/2-motivation/circle/mot-cercle-dis-2.pdf}
        \end{figure}
      }
      \only<5,6>
      {
        \begin{figure}[h!]
          \centering
          \includegraphics[width=4cm]{fig/2-motivation/circle/mot-cercle-dis-3.pdf}
        \end{figure}
      }          
      \end{column}
      \begin{column}{5cm}
        \vspace{-0.4cm}
        
        \only<3>
        {
          \begin{block}{Remarques}
            1. On peut facilement connaître la position d'un point par rapport au disque.\\
            2. La définition n'est pas suffisante pour représenter tous les bords de disques discrets.
          \end{block}
        }
        \only<4,5>
        {
          \begin{block}{Remarques}
            Les points sont ordonnées sur une structure régulière.
          \end{block}
        }
        \only<5,6>
        {
          \begin{block}{Conséquences}
            Se concentrer sur l'étude des points du bords.
          \end{block}
        }
        \only<6>
        {
          \begin{block}{Question :}
            \begin{center}  
              \alert{Comment sont représentés les bords des disques discrets dans $\mathbb{Z}^{2}$ ?}
            \end{center}
          \end{block}
        }
      \end{column}
    \end{columns}
\end{frame}

%-----------------------------------------------------------------
\subsection{Alpha-Shape}
%-----------------------------------------------------------------

\begin{frame}
  \frametitle{Un outil, l'$\alpha$-shape}
  \begin{block}{Définition}
    \begin{itemize}
      \item $\alpha$-hull : Intersection de tous les disques généralisés de rayon $1/\alpha$ qui contiennent tous les points de l'ensemble.
      \item Point $\alpha$-extreme : Point appartenant au bord de l'$\alpha$-hull.
      \item $\alpha$-shape : Enveloppe reliant tous les $\alpha$-extremes adjacents.
    \end{itemize}
\begin{tiny}
  [EKS83] Edelsbrunner, H., Kirkpatrick, D., Seidel, R.\\
  On the Shape of a Set of Points in the Plane\\
  {\em IEEE Transactions on Information Theory}, 29(4):551--559, 1983.\\
\end{tiny}   
  \end{block}
  \vspace{-1cm}
  \only<1>
  { 
    \begin{columns}[t]
      \begin{column}{5cm}
        \begin{figure}[h!]
          \centering
          \includegraphics[width=\linewidth,page=1]{fig/2-motivation/mot-alpha-shape.pdf}
         \end{figure}
       \end{column}
       \begin{column}{5cm}
         \begin{figure}[h!]
           \centering
           \includegraphics[width=\linewidth,page=3]{fig/2-motivation/mot-alpha-shape.pdf}
         \end{figure}
       \end{column}
    \end{columns} 
  }
  \only<2>
  { 
    \begin{columns}[t]
      \begin{column}{5cm}
        \begin{figure}[h!]
          \centering
          \includegraphics[width=\linewidth,page=2]{fig/2-motivation/mot-alpha-shape.pdf}
         \end{figure}
       \end{column}
       \begin{column}{5cm}
         \begin{figure}[h!]
           \centering
           \includegraphics[width=\linewidth,page=4]{fig/2-motivation/mot-alpha-shape.pdf}
         \end{figure}
       \end{column}
    \end{columns} 
  }
\end{frame}

%-----------------------------------------------------------------

\begin{frame}
  \frametitle{Les $\alpha$-shapes de disques discrets}
 
  \begin{columns}[t]
    \begin{column}{4cm}
      \begin{exampleblock}{Cercle}
        $\mathcal{C} \left( (24,-8), (16,1), (0,0) \right)$\\
         
        \only<1>
        {
          Centre : (8.86, -13.4)\\
          R : 16.06
        }
        \only<2>
        {
          $\alpha = -\sqrt{2}$\\
           Nb Sommets : 126\\
        }
        \only<3>
        {
          $\alpha = -1$\\
          Nb Sommets : 89\\
        }
        \only<4>
        {
          $\alpha = -1/\sqrt{10}$\\
           Nb Sommets : 71\\
        }
        \only<5>
        {
          $\alpha = -1/\sqrt{20}$\\
           Nb Sommets : 64\\
        }
        \only<6>
        {
          $\alpha = -1/\sqrt{80}$\\
           Nb Sommets : 61\\
        }
        \only<7>
        {
          $\alpha = 0$\\
           Nb Sommets : 23\\
        }
        \only<8>
        {
          $\alpha = 1/\sqrt{600}$\\
           Nb Sommets : 21\\
        }
        \only<9>
        {
          $\alpha = 1/\sqrt{400}$\\
           Nb Sommets : 19\\
        }
        \only<10>
        {
          $\alpha = 1/\sqrt{300}$\\
           Nb Sommets : 11\\
        }
        \only<11>
        {
          $\alpha = 1/\sqrt{260}$\\
           Nb Sommets : 6\\
        }
      \end{exampleblock}
     
      \only<12>
      {
        \begin{block}{}
          Les $\alpha$-shapes récupèrent un beau panel de distribution de points composants le bord de disques discrets.
        \end{block}
      }
      
    \end{column}

    \begin{column}{6cm}
      \vspace{-0.8cm}     
      \only<1>
      {
        \begin{figure}[h!]
          \centering
          \includegraphics[width=5.5cm]{fig/2-motivation/alpha-shape-circle/as-0.pdf}
        \end{figure}
      }
      \only<2>
      {
        \begin{figure}[h!]
          \centering
          \includegraphics[width=5.5cm]{fig/2-motivation/alpha-shape-circle/as-1.pdf}
        \end{figure}
      }
      \only<3>
      {
        \begin{figure}[h!]
          \centering
          \includegraphics[width=5.5cm]{fig/2-motivation/alpha-shape-circle/as-2.pdf}
        \end{figure}
      }
      \only<4>
      {
        \begin{figure}[h!]
          \centering
          \includegraphics[width=5.5cm]{fig/2-motivation/alpha-shape-circle/as-3.pdf}
        \end{figure}
      }
      \only<5>
      {
        \begin{figure}[h!]
          \centering
          \includegraphics[width=5.5cm]{fig/2-motivation/alpha-shape-circle/as-4.pdf}
        \end{figure}
      }
      \only<6>
      {
        \begin{figure}[h!]
          \centering
          \includegraphics[width=5.5cm]{fig/2-motivation/alpha-shape-circle/as-5.pdf}
        \end{figure}
      }
      \only<7>
      {
        \begin{figure}[h!]
          \centering
          \includegraphics[width=5.5cm]{fig/2-motivation/alpha-shape-circle/as-6.pdf}
        \end{figure}
      }
      \only<8>
      {
        \begin{figure}[h!]
          \centering
          \includegraphics[width=5.5cm]{fig/2-motivation/alpha-shape-circle/as-7.pdf}
        \end{figure}
      }
      \only<9>
      {
        \begin{figure}[h!]
          \centering
          \includegraphics[width=5.5cm]{fig/2-motivation/alpha-shape-circle/as-8.pdf}
        \end{figure}
      }
      \only<10>
      {
        \begin{figure}[h!]
          \centering
          \includegraphics[width=5.5cm]{fig/2-motivation/alpha-shape-circle/as-9.pdf}
        \end{figure}
      }
      \only<11,12>
      {
        \begin{figure}[h!]
          \centering
          \includegraphics[width=5.5cm]{fig/2-motivation/alpha-shape-circle/as-10.pdf}
        \end{figure}
      }

    \end{column}
  \end{columns}
\end{frame}

%-----------------------------------------------------------------

\begin{frame}
  \frametitle{Remarques}
  \only<1>
  {
    \begin{columns}[t]
      \begin{column}{5cm}
        \begin{figure}[h!]
          \centering
          \includegraphics[width=\linewidth,page=8]{fig/2-motivation/circle/mot-union-neg.pdf}
         \end{figure}
       \end{column}
       \begin{column}{5cm}
         \begin{figure}[h!]
           \centering
           \includegraphics[width=\linewidth,page=2]{fig/2-motivation/circle/mot-union-pos.pdf}
         \end{figure}
       \end{column}
    \end{columns}
    
    \begin{block}{}
      La construction d'une $\alpha$-shape s'appuie sur les triangulations de Delaunay.    
    \end{block}
  }
  \only<2>
  {
    \begin{columns}[t]
      \begin{column}{5cm}
        \begin{figure}[h!]
          \centering
          \includegraphics[width=\linewidth]{fig/2-motivation/alpha-shape-circle/as-1.pdf}
         \end{figure}
       \end{column}
       \begin{column}{5cm}
         \begin{figure}[h!]
           \centering
           \includegraphics[width=\linewidth]{fig/2-motivation/alpha-shape-circle/as-2.pdf}
         \end{figure}
       \end{column}
    \end{columns}
    
    \begin{block}{Suivi de bord}
      Pour $\alpha = -\sqrt{2}$ et $\alpha = -1$ on réalise un suivi de bord 4/8-connexes.   
    \end{block}
  }
\end{frame}



%------------------------------------------------
\section{Éxistants}
%------------------------------------------------

\begin{frame}
\frametitle{Suivi de bord}

\end{frame}

%-----------------------------------------------------------------
\subsection{Enveloppe Convexe}
%-----------------------------------------------------------------

\begin{frame}
\frametitle{Har Peled - Convergents}

\end{frame}



\begin{frame}
\frametitle{Har Peled - Enveloppe Convexe de Disques}

\end{frame}

\begin{frame}
\frametitle{Algorithme}

\end{frame}

\begin{frame}
\frametitle{Résultats}

\end{frame}


%------------------------------------------------
\section{Contributions}
%------------------------------------------------


\begin{frame}
\frametitle{Introduction}
\begin{block}{Indices d'existance}
% indiquant la possible existance d'un algorithme incrémental et output sensitive calculant les $\alpha$-shapes de la discrétisation d'un disque (quand $\alpha$ est négatif) : 

\begin{enumerate}
\item Le bord de la discrétisation d'un disque est convexe. Tout bord convexe se décompose de manière unique en motifs de droite discrète \cite{roussillonPR2011}. 
\item Il existe un algorithme incrémental et output sensitive calculant les motifs de la discrétisation d'un disque par les convergents. \cite{HarPeled98}
\item L'union des alpha-shapes lorsque $\alpha$ est négatif est un sous-ensemble de la triangulation de Delaunay \cite{EdeKirSei83}et la triangulation de Delaunay d'un motif est déterminé par les convergents de sa pente \cite{RoussillonL11}. 
\end{enumerate}

\end{block}

%----- Begin biblio -----
\begin{columns}[t]
  \begin{column}{0.5\linewidth}
    \scriptsize  
    \begin{thebibliography}{alpha}
      \bibitem{roussillonPR2011}
      [RS11] T. Roussillon and I. Sivignon
      \newblock Faithful Polygonal Representation of the Convex and Concave Parts of a Digital Curve
      \newblock {\em Pattern Recognition}, 2011.
      
      \bibitem{HarPeled98}
      [HP98] Har-Peled
      \newblock An Output Sensitive Algorithm for Discrete Convex Hulls
      \newblock {\em ICGTA: Computational Geometry: Theory and Applications}, 1998.  
    \end{thebibliography}
    \scriptsize
  \end{column}
  \begin{column}{0.5\linewidth}
    \scriptsize
    \begin{thebibliography}{alpha}
      \bibitem{EKS83}
      [EKS83] Edelsbrunner, H., Kirkpatrick, D., Seidel, R.
      \newblock On the Shape of a Set of Points in the Plane
      \newblock {\em IEEE Transactions on Information Theory}, 29(4):551--559, 1983.
      
      \bibitem{RoussillonL11}
      [RL11] Tristan Roussillon and Jacques-Olivier Lachaud
      \newblock Delaunay Properties of Digital Straight Segments
      \newblock {\em Springer : Discrete Geometry for Computer Imagery - 16th {IAPR}}, 2011.
     \end{thebibliography}
    \scriptsize     
  \end{column}
\end{columns}  
%----- End biblio   -----

\end{frame}



\begin{frame}
\frametitle{Relations aux triangulations de Delaunay}

\end{frame}

%-----------------------------------------------------------------
\subsection{$\alpha$-shape, $\alpha \leq 0$ - Généralisation de Har-Peled}
%-----------------------------------------------------------------

\begin{frame}
\frametitle{$\alpha$-shape, $\alpha \leq 0$ - Généralisation de Har-Peled - Présentation}

\end{frame}

\begin{frame}
\frametitle{Algorithme}

\end{frame}

\begin{frame}
\frametitle{Résultats}

\end{frame}

%-----------------------------------------------------------------
\subsection{$\alpha$-shape, $\alpha \leq 0$}
%-----------------------------------------------------------------

\begin{frame}
\frametitle{$\alpha$-shape, $\alpha \leq 0$ - Présentation}

\end{frame}

\begin{frame}
\frametitle{Algorithme}

\end{frame}

\begin{frame}
\frametitle{Résultats}

\end{frame}




%-----------------------------------------------------------------
\section{Poursuite}
%-----------------------------------------------------------------

%-----------------------------------------------------------------
\subsection{Poursuite du projet}
%-----------------------------------------------------------------

\begin{frame}
\frametitle{Poursuite du projet}

  \begin{block}{Amélioration}
    \begin{itemize}
      \item Approche ascendante, top-down pour le calcul de l'$\alpha$ shape quand $\alpha > 0$.
    \end{itemize}
  \end{block}
  \begin{block}{Augmenter les points de comparaison}
    \begin{itemize}
      \item Approche récursive pour $\alpha <0$ basée sur une décomposition par arête.
      \item Utiliser d'autres formes discrètes : Ellipse.
    \end{itemize}
  \end{block}
  \begin{block}{Utilisation des algorithmes}
    \begin{itemize}
      \item Utiliser les algorithmes Output Sensitive pour la reconnaissance de disques discrets.
    \end{itemize}
  \end{block}


\end{frame}

%-----------------------------------------------------------------
\subsection{Poursuite personnelle}
%-----------------------------------------------------------------

\begin{frame}
\frametitle{Poursuite personnelle}
  \begin{block}{Domaine scientifique}
    \begin{itemize}
			\item Mathématiques appliquées.
			\item Informatiques.
		\end{itemize}	
  \end{block} 
  
    \begin{block}{Contexte}
    \begin{itemize}
      \item Travaux de recherche.
			\item Intérêt pour l'enseignement.
		\end{itemize}	
  \end{block} 
  
  \begin{exampleblock}{}
    Langage C++.
  \end{exampleblock} 
\end{frame}

\subsection{Thanks}
\begin{frame}
  \begin{block}{}
    \begin{center}
      Merci !
    \end{center}

  \end{block}
\end{frame}


%----------------------------------------------------------------------------------------
\end{document} 
