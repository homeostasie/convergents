%%%%%%%%%%%%%%%%%%%%% 1-introduction.tex %%%%%%%%%%%%%%%%%%%%%%%%%%%%%%%%%
%
% Introduction
%
% ... and motivation
%
%%%%%%%%%%%%%%%%%%%%%%%% Springer-Verlag %%%%%%%%%%%%%%%%%%%%%%%%%%

\chapter{Introduction}
\label{pt1-ch1-intro} % Always give a unique label
% use \chaptermark{}
% to alter or adjust the chapter heading in the running head

Your text goes here. Separate text sections with the standard \LaTeX\
sectioning commands.

\section{Contexte pédagogique}
\label{pt1-ch1-sec:1}

% Parcours - ok
Dans le cadre de ma deuxième année en master professionnel de Mathématiques : Statistiques, Informatique et Techniques Numériques (SITN) à l’Université Claude Bernard - Lyon 1, je me dois de réaliser un stage dans le but de valider ma formation et d'ainsi obtenir mon diplôme. \newline
%labo -ok
Je suis actuellement accueilli pour une durée de 6 mois au LIRIS - Laboratoire d'InfoRmatique en Image et Systèmes d'information - UMR 5205 CNRS. Le LIRIS est issue de la fusion de plusieurs pôles de recherche de la région Lyonnaise. Aujourd'hui composé d'environ 300 personnes, il participe activement à la recherche et l'éducation à travers ces deux grands départements thématiques : "Image" et "Données, Connaissances, Services".\newline

%encadrant
Je suis encadré par Tristan Roussillon, Maitre de conférence à l'insa et membre de l'équipe M2Disco - Modèles Multirésolution, Discrèts et Combinatoires. \newline

%l'équipe & transition
L'équipe M2Disco fait partie (?? e ou pas e ??) du département Image du LIRIS. Tout au long de mon stage, elle n'a eu de cesse de vouloir m'imprégner de sa problématique d'étude à la croisée de plusieurs domaines de l’informatique et des mathématiques. Elle m'a ainsi motivé à travailler sur des méthodes issues de domaine (?? s ou pas s??) tels que l'optimisation combinatoire, la géométrie algorithmique et la géométrie discrète à travers l'étude et la modélisation de contours discrets.\newline

% plus classique
% L'équipe M2Disco fait partie (?? e ou pas e ??) du département Image du LIRIS. Elle compose avec des problèmes à la croisée de plusieurs domaines de l’informatique et des mathématiques et travaille sur des domaines tels que l'optimisation combinatoire, la géométrie algorithmique et la géométrie discrète.



\section{Motivations scientifiques}
\label{pt1-ch1-sec:2}

La problématique de mon stage s'intéresse à l'organisation de la structure du bord d'un cercle discret. En effet, cet ensemble de sommets et d'arêtes qui sépare l'intérieur de l'extérieur du cercle semble respecter une certaine régularité, une certaine consistance géométrie.
Néanmoins, avant ce rentrer dans le vif du sujet, il convient d'introduire plus en détail le domaine duquel il est issu (?? e ou pas e??) : la géométrie discrète.


\subsection{Géométrie discrète}
\label{1.1-geodiscete.png}

La géométrie discrète, également appelée digital geometry dans la langue Anglaise est un domaine particulièrement vaste qui se situe à l'intersection de bon nombre de domaines scientifiques.


\includegraphics[height=6cm]{pics/1-1_geodiscete.png}
(?? crédit, centrée, légendée, passé au latex pur??)



\subsection{Du disque Euclidien au cercle discret}
\label{pt1-ch1-sec:2.2}

\subsection{Organisation des points d'un cercle discret}
\label{pt1-ch1-sec:2.3}

\subsection{Représentation cercle discret}
\label{pt1-ch1-sec:2.4}

\subsection{Jeté de rayon et convergents}
\label{pt1-ch1-sec:2.5}

\subsection{$\alpha$-shape, enveloppe convexe et triangulation}
\label{pt1-ch1-sec:2.6}


On s'intéresse au problème connu et ouvert de l'organisation du bord d'un cercle discret. Cette organisation semble respecter une certaine régularité et ainsi ne pas être totalement aléatoire. Ce domaine communique d'ailleurs avec bien d'autres domaines des Mathématiques en traitant des Entiers de Gauss. \newline

Pour approcher ce problème, le sujet du stage propose d'étudier et de regarder les alpha-shapes de cercle discret. \newline

---> dire pourquoi, c'est intéressant.  \newline
---> potentiellement mettre des images. \newline

Une variation de alpha implique une récupération plus ou moins précises du bord. Alpha = 0 --> enveloppe convexe. Pour Alpha < -1 --> tous les points au bord et à l'intérieur, Alpha > 1 -->  tous les points au bord et à l'extérieur, on parle alors de suivi de contour. Implémenter dans la class Tracking. \newline




 
