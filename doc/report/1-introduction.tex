%------------------------------------------------
\section{Introduction}
%------------------------------------------------

%-----------------------------------------------------------------
\subsection{Contexte pédagogique}
%-----------------------------------------------------------------

% Parcours - ok
Actuellement en deuxième année du master professionnel de Mathématiques : Statistiques, Informatique et Techniques Numériques (SITN) à l’Université Claude Bernard - Lyon 1, je réalise un stage dans le but de valider ma formation et d'ainsi obtenir mon diplôme. \newline

%labo -ok
Je suis actuellement accueilli pour une durée de 6 mois au LIRIS - Laboratoire d'InfoRmatique en Image et Systèmes d'information - UMR 5205 CNRS. Le LIRIS est issue de la fusion de plusieurs pôles de recherche de la région Lyonnaise. Aujourd'hui composé d'environ 300 personnes, il participe activement à la recherche et à l'éducation à travers deux grands départements thématiques : "Image" et "Données, Connaissances, Services".\newline

%encadrant
Je suis encadré par Tristan Roussillon, Maitre de conférence à l'Insa - Institut National des Sciences Appliquées et membre de l'équipe M2Disco - Modèles Multirésolution, Discrets et Combinatoires. Cette équipe est une composante du département Image du LIRIS qui traite de sujets comme l'analyse d'images, l'optimisation, la programmation par contraintes et la géométrie discrète. C'est d'ailleurs sur cette dernière branche que le contexte de mon stage s'est déroulé.



%-----------------------------------------------------------------
\subsection{Contexte scientifique}
%-----------------------------------------------------------------

J'ai découvert pour l’occasion la géométrie discrète, plus fréquemment appelé digital geometry. Cette discipline de recherche se trouve à l'intersection de bien des domaines mathématiques et informatiques.\\

%(pics de digital geometry) % traduire ?
   
Il s'agit principalement d'étudier la géométrie et la topologie de formes, d'objets portés sur des structures régulières. Dans le cadre de ce stage, nous intéresserons plus particulièrement au disque discret sur la grille $\mathbb{Z}^{2}$. 
