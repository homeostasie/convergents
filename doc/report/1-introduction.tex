%------------------------------------------------
\section{Introduction}
%------------------------------------------------

%-----------------------------------------------------------------
\subsection{Contexte pédagogique}
%-----------------------------------------------------------------

% Parcours - ok
Actuellement en deuxième année du master professionnel de Mathématiques : Statistiques, Informatique et Techniques Numériques (SITN) à l’Université Claude Bernard - Lyon 1, je réalise un stage dans le but d'enrichir mes connaissances dans le domaine scientifique, de découvrir le monde professionnel et ainsi de valider ma formation par l'obtention d'un diplôme.\newline

%labo -ok
Je suis accueilli pour une durée de six mois au LIRIS - Laboratoire d'InfoRmatique en Image et Systèmes d'information - UMR 5205 CNRS. Le LIRIS est issu de la fusion de plusieurs pôles de recherche de la région Lyonnaise. Aujourd'hui composé d'environ 300 personnes, il participe activement à la recherche 
%et à l'éducation 
à travers deux grands départements thématiques : "Image" et "Données, Connaissances, Services".\newline

%encadrant
Je suis encadré par Tristan Roussillon, Maître de conférence à l'Insa - Institut National des Sciences Appliquées et membre de l'équipe M2Disco - Modèles Multirésolution, Discrets et Combinatoires. Cette équipe est une composante du département Image du LIRIS qui traite de sujets comme l'analyse d'images, l'optimisation, la programmation par contraintes et la géométrie discrète. C'est d'ailleurs sur cette dernière branche que le contexte de mon stage s'est déroulé.

%-----------------------------------------------------------------
\subsection{Contexte scientifique}
%-----------------------------------------------------------------

À l'occassion de ce stage, j'ai pu découvrir la géométrie discrète, appelée ``discrete geometry'' ou ``digital geometry'' en anglais. Cette discipline de recherche se trouve à l'intersection de plusieurs domaines des mathématiques et de l'informatiques: topologie, arithmétique, combinatoire, géométrie algorithmique, programmation linéaire. \\
 
Il s'agit principalement d'étudier la géométrie et la topologie d'objets portés sur des structures régulières. Dans ce stage, nous étudierons le disque discret, défini comme l'intersection entre un disque Euclidien et la grille $\mathbb{Z}^{2}$. 
