%------------------------------------------------
\section{Motivations}
%------------------------------------------------

%-----------------------------------------------------------------
\subsection{Le bord du disque discret}
%-----------------------------------------------------------------

%-----------------------------------------------------------------
\subsubsection{Du disque Euclidien au disque discret}

Avant de s'attarder sur la représentation d'un disque discret, il convient de revenir un court instant au disque fermé Euclidien et à l'ensemble reprenant tous les points de son bord, le cercle Euclidien.

\begin{Definition}{Disque fermé Euclidien}
\label{def:disk-euc}
 $$\mathcal{D}_e =  \left\{ (x,y) \in \mathbb{R}^{2} |  (x - u)^2 + (y - v)^2 \leq R^2. \right\}$$
\end{Definition}

\begin{Definition}{Cercle Euclidien}
\label{def:cer-euc}
  $$\mathcal{C}_e =  \left\{ (x,y) \in \mathbb{R}^{2} |  (x - u)^2 + (y - v)^2 = R^2. \right\}$$
\end{Definition}

avec $(u,v) \in \mathbb{R}^{2}$ les coordonnées du centre et $R \in \mathbb{R}^{+*}$ le rayon.\\

%PICS et PICS de disque et de cercle euclidien

Si la définition du disque fermé discret dans $\mathbb{R}^{2}$ (appelé pour la suite du rapport uniquement disque discret) reste très proche de la définition Euclidienne, l'équivalent discret pour récupérer tous les points du bord quant à elle change considérablement.

\begin{Definition}{Disque discret}
\label{def:disk-dis}
$$\mathcal{D} =  \left\{ (x,y) \in \mathbb{Z}^{2} |  ax + by + c(x^2 + y^2) + d \leq 0. \right\}$$
\end{Definition}

avec $(u,v) \in \mathbb{R}^{2}$ les coordonnées du centre et $R \in \mathbb{R}^{+*}$ le rayon.\\

%PICS de disque discret

L'ensemble des points appartenant à $\mathbb{Z}^{2}$ représentés en rouge sur le dessin et étant positionnés exactement sur le cercle euclidien ne suffisent pas à représenté l'ensemble des points du cercle discret. Il faut donc prendre une définition plus générale.

\begin{Definition}{Cercle discret}
\label{def:cer-dis}
  $$ \mathcal{C} =  \left\{ (x,y) \in \mathbb{Z}^{2} | \exists (i,j) \in \{0,1\}^2, (i,j) \ne (0,0) | (x+i,y+j) \notin \mathcal{D} \right\}$$
\end{Definition}


%PICS de cercle discret

%-----------------------------------------------------------------
\subsubsection{L'étude des points du bord}

En continuant l'investigation de la définition précédente du cercle discret, il est possible de formuler plusieurs remarques. De part le fait de travailler sur $\mathbb{Z}^{2}$, on remarque que les points sont ordonnées sur une grille régulière. Pour un point donnée, il est alors possible d'atteindre quatre voisins en suivant les quatre directions cardinales d'une unité. En éffet, chaque point possède un voisin, en haut, à gauche, en bas et à droite de lui. \\

%PICS de réseau 4-connexes

Or, on observe que les points strictement à l'intérieur du disque discret (et donc pas sur le bord) possède tous leurs quatre voisins également à l'intérieur du disque. Pour autant, les points du bords du disque discret possèdent eux une distribution différente. Chacun des points du bord possède entre un et trois voisins à l'extérieur du disque et le complémentaire à quatre à l'intérieur.\\

PICS de points du bord sans 4 voisins in ??\\

Pour définir un disque discret, on se retrouve finalement avec deux ensembles. Le premier représente les points strictement à l'intérieur. Ils sont ordonnées avec exactement quatre voisins également à l'intérieur du disque. Le deuxième ensemble égale au cercle discret propose une distribution de points bien moins régulière. Finalement, seule l'étude des points du bord nous semble pertinente pour comprendre l'organisation et la structure des disque discret.


%-----------------------------------------------------------------
\subsection{Alpha-Shape}
%-----------------------------------------------------------------

%-----------------------------------------------------------------
\subsubsection{Définition générale}

En s'intéressant principalement à des contours de formes discrètes, un ensemble d'outil nous est apparu comme particulièrement opportun. Il s'agit des $\alpha$-hull et des $\alpha$-shape définit pour la première fois par edel (référence requise) de la manière suivante.

\begin{Definition}{$\alpha$-hull de $\mathcal{S}$}
\label{def:ah}
    $$\left\{ \cap \mathcal{D}_{\alpha} | \forall (x,y)\in \mathcal{S} \subset \mathbb{Z}^{2} \Rightarrow (x,y) \in \mathcal{D}_{\alpha} \right\}$$
    Intersection de tous les disques généralisés de rayon $1/\alpha$ qui contiennent tous les points de l'ensemble.
\end{Definition}

%PICS avec les alpha hull

Les sommets de $\alpha$-hull, les points appartenant à son bord de l'$\alpha$-hull sont appelés points $\alpha$-extreme. S'ils sont situés sur le même bord en étant relié par un arc de cercle de rayon 1/ $\lvert \alpha \rvert$, on dit alors qu'ils sont adjacents.

\begin{Definition}{$\alpha$-shape}\\
\label{def:as}
      Enveloppe reliant tous les $\alpha$-extremes adjacents.
\end{Definition}

%PICS avec les alpha-shape\\

L'utilité des $\alpha$-shapes est d'être une sous-ensemble des points du bord. 
% need more

%-----------------------------------------------------------------
\subsubsection{$\alpha$-shapes de disque discret}

En prenant un panel assez large d'$\alpha$-shape avec $\alpha$ succéssivement négatif et positif variant de -2 à $R_D$ (le rayon du disque) on remarque que les $\alpha$-shapes représentent un large panel d'ensemble représentatif des points du bord d'un disque discret.

%PICS avec alpha shapes de disque.

%-----------------------------------------------------------------
\subsubsection{Plusieurs remarques}

\paragraph{}
Le cas central de $\alpha = 0$ représente une intersection de disque de rayon infini, cela peut être interprété comme une intersection de demi-plan. On retrouve exactement le calcul de l'enveloppe convexe.

\paragraph{}
Les plus petits $\alpha = -2$ et $\alpha = -1$ représenter les cas bien connus de suivi de bord. En effet, en ne pouvant s'éloigner au plus de disques de rayon -2 et -1, on ne peut suivre le bord de nos disques que par l'intermédiaire des voisins 4-connexes et 8-connexes de chaque point du bord.

\paragraph{}
En réalisant l'union de nos $\alpha$-shape pour le cas négatif et le cas positif, il semblerait que l'on s'appuie également sur les triangulations d'ordre 0 et n de Delaunay. Mais nous reviendrons dessus plus en détail dans la section "Existants".\\

% need more 
