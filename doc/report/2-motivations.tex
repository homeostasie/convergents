%------------------------------------------------
\section{Motivations}
%------------------------------------------------

%-----------------------------------------------------------------
\subsection{Le bord du disque discret}
%-----------------------------------------------------------------

%-----------------------------------------------------------------
\subsubsection{Du disque Euclidien au disque discret}


Le disque discret $\mathcal{D}$ est défini par analogie avec le disque Euclidien $\mathcal{D}_e$ qui représente l'ensemble des points situé à une distance inférieure à $R$ de son centre $O(u,v)$.


\begin{Definition}{Disque fermé Euclidien}
\label{def:disk-euc}
 $$\mathcal{D}_e =  \left\{ (x,y) \in \mathbb{R}^{2} |  (x - u)^2 + (y - v)^2 \leq R^2 \right\}$$
\end{Definition}

\begin{Definition}{Disque fermé discret}
\label{def:disk-dis}
  $$\mathcal{D} =  \left\{ (x,y) \in \mathbb{Z}^{2} |  (x - u)^2 + (y - v)^2 \leq R^2. \right\}$$
  
  avec $(u,v) \in \mathbb{Q}^{2}$ les coordonnées du centre et $R^2 \in \mathbb{Q}$ le rayon.\\
\end{Definition}

\begin{figure}[H]
  \centering
  \includegraphics[width=6cm]{fig/2-mot/circle/circle-euc-0.pdf}
  \includegraphics[width=6cm]{fig/2-mot/circle/circle-dis-0.pdf}
  \caption{Disque Euclidiens et Discret}
\end{figure}

La définition du disque fermé discret dans $\mathbb{Z}^{2}$ (appelé pour la suite du rapport uniquement disque discret) reste très proche de la définition Euclidienne. De plus, la définition du cercle Euclidien découle directement de la définition du disque Euclidien.

Néanmoins, la définition du cercle discret ne dérive pas explicitement de la définition du disque discret. En effet, l'ensemble de points, représentés en rouge sur le dessin et appartenant à l'ensemble : $\left\{ (x,y) \in \mathbb{Z}^{2} |  (x - u)^2 + (y - v)^2 = R^2 \right\}$ ne suffit pas à représenter le cercle discret. Pour étudier le cercle discret, nous avons besoin de définir la notion de voisinage et de revenir à une définition plus générale du bord de toute ensemble discret.\\

\begin{Definition}{4-Voisinage d'un point $(u,v)$}
\label{def:vois-4}
  $$\mathcal{V}_4(u,v) =  \left\{ (x,y) \in \mathbb{Z}^{2} |  |x-u|+|y-u| = 1 \right\}$$
\end{Definition}

\begin{Definition}{8-Voisinage d'un point $(u,v)$}
\label{def:vois-8}
  $$\mathcal{V}_8(u,v) =  \left\{ (x,y) \in \mathbb{Z}^{2} |  max(|x-u|,|y-u|) = 1 \right\}$$
\end{Definition}

\begin{figure}[H]
  \centering
  \includegraphics[width=.6\linewidth]{fig/2-mot/connexe/connexite.pdf}
  \caption{4-connexité et 8-connexité}
\end{figure}

En définissant les points du bord du disque discret comme l'ensemble des points appartenant au disque dont le 4-voisinage n'est pas intégralement contenu dans le disque, on définit un ensemble 8-connexes alors que respectivement en choisissant un 8-voisinage, on obtient un ensemble 4-connexe.

\begin{Definition}{Bord 8-connexes (*=4) et 4-connexes (*=8) d'un ensemble discret $\mathcal{S}$}
\label{def:bord-ens}
  $$ \partial \mathcal{S}_{*} =  \left\{ (x,y) \in \mathcal{S} | \left( \mathcal{V}_{*}(x,y) \cap \mathcal{S} \right) \neq \mathcal{V}_{*}(x,y) \right\}$$
\end{Definition}

On revient à la définition du cercle discret représentant le bord d'un disque discret.

\begin{Definition}{Cercle discret}
\label{def:cer-dis}
  $$ \mathcal{C}_{*} =  \left\{ (x,y) \in \mathcal{D} | \left( \mathcal{V}_{*}(x,y) \cap \mathcal{D} \right) \neq \mathcal{V}_{*}(x,y) \right\}$$
\end{Definition}

\begin{figure}[H]
  \centering
  \includegraphics[width=6cm]{fig/2-mot/circle/circle-dis-1a.pdf}
  \includegraphics[width=6cm]{fig/2-mot/circle/circle-dis-1b.pdf}
  \caption{Cercle Discret 8-connexes et 4-connexes}
\end{figure}


%-----------------------------------------------------------------
\subsubsection{Énoncé de la problématique}

Les points sont organisés sur une grille régulière : $\mathbb{Z}^{2}$. Pour chaque point, il est possible d'atteindre quatre voisins en se dirigeant d'une unité vers les quatre points cardinaux. Chaque point possède donc un voisin, en haut, à gauche, en bas et à droite de lui. Or, on observe que les points strictement à l'intérieur du disque discret et non sur le bord (en bleu clair) possèdent tous leurs quatre plus proches voisins à l'intérieur du disque (maillage en rouge).

\begin{Definition}{Ensemble de points strictement à l'intérieur d'un disque}
\label{def:int-ens}
  $$\stackrel{\ \circ}{\mathcal{D}}_{*} = \mathcal{D} \ \mathcal{C_{*}} $$
  $$ \stackrel{\ \circ}{\mathcal{D}}_{*} =  \left\{ (x,y) \in \mathcal{D} | \mathcal{N}_{*}(x,y) \cap \mathcal{D} = \mathcal{N}_{*}(x,y) \right\}$$
\end{Definition}

\begin{figure}[H]
  \centering
  \includegraphics[width=6cm]{fig/2-mot/circle/circle-dis-2.pdf}
  \caption{Réseau de point à strictement à l'intérieur.}
\end{figure}

On se retrouve alors avec deux ensembles disjoints pour définir et représenter un disque discret. Le premier est bien organisé, il représente les points strictement à l'intérieur du disque. Le deuxième ensemble représente les points du bord du disque et son organisation est moins triviale. 

De part la régularité du premier, seule l'étude des points du bord nous semble pertinente pour comprendre l'organisation et la structure des disque discret.\\

Nous allons chercher à comprendre comment est organisée la structure des points du bord d'un disque discret.


%-----------------------------------------------------------------
\subsection{Alpha-Shape}
%-----------------------------------------------------------------

En s'intéressant principalement à des contours de formes discrètes, un ensemble d'outil nous est apparu comme particulièrement opportun. Il s'agit des $\alpha$-hull et des $\alpha$-shape définit pour la première fois par Edelsbrunner \cite{EdeKirSei83} et faisant appel à des disques généralisés.\\

Un disque généralisé permet de définir des disques avec des rayons négatifs en faisant appel au complémentaire.

\begin{Definition}{Disques généralisés de rayon $1/\alpha$}\\
\label{def:dis-gen}
   $\mathcal{D}_{\alpha}$ est le disque fermé de rayon $1/\alpha$ pour $\alpha > 0$.\\
   $\mathcal{D}_{\alpha}$ est le complémentaire fermé du disque de rayon $- 1/\alpha$ pour $\alpha < 0$.
\end{Definition}

%-----------------------------------------------------------------
\subsubsection{Définition}

\begin{Definition}{$\alpha$-hull de $\mathcal{S}$}\\
\label{def:ah-txt}
    Intersection de tous les disques généralisés de rayon $1/\alpha$ qui contiennent tous les points de l'ensemble.
    $$\cap \left\{ \mathcal{D}_{\alpha} | \mathcal{S} \in \mathcal{D}_{\alpha} \right\}$$
\end{Definition}

\begin{figure}[h!]
  \centering
  \includegraphics[width=0.4\linewidth,page=1]{fig/2-mot/as/mot-alpha-shape.pdf}
  \includegraphics[width=0.4\linewidth,page=3]{fig/2-mot/as/mot-alpha-shape.pdf}
  \caption{$\alpha$-Hull négative et $\alpha$-Hull positive }
\end{figure}
  

Les sommets de $\alpha$-hull sont appelés points $\alpha$-extrêmes. S'ils sont reliés par un arc de cercle de rayon 1/ $\lvert \alpha \rvert$ qui n'exclut pas de points, on dit alors qu'ils sont adjacents.

\begin{Definition}{$\alpha$-shape}\\
\label{def:as}
      Graphe plongé dans le plan reliant tous les points $\alpha$-extrêmes adjacents par des segments de droite.
\end{Definition}

\begin{figure}[h!]
  \centering
  \includegraphics[width=0.4\linewidth,page=2]{fig/2-mot/as/mot-alpha-shape.pdf}
  \includegraphics[width=0.4\linewidth,page=4]{fig/2-mot/as/mot-alpha-shape.pdf}
  \caption{$\alpha$-Shape négative et $\alpha$-Shape positive }
\end{figure}


%-----------------------------------------------------------------
\subsubsection{$\alpha$-shapes de disque discret}

En choisissant successivement des valeurs de $\alpha$ variant de -2 à $R_D$ (le rayon du disque discret), on calcul un vaste ensemble d'$\alpha$-shape. on remarque que ses $\alpha$-shapes représentent à chaque fois un sous-ensemble des points du bord d'un disque discret.

%PICS avec alpha shapes de disque.

%-----------------------------------------------------------------
\subsubsection{Propriétés}

\paragraph{}
Le cas central lorsque $\alpha = 0$ représente une intersection de disque de rayon infini qui est interprété comme une intersection de demi-plan. On obtient une enveloppe convexe.

\paragraph{}
Les plus petits cas où $\alpha = -2$ et $\alpha = -\sqrt{2}$ permettent de récupérer exactement les points du bords dans le cas 4-connexes et 8-connexes. En effet, en ne pouvant s'éloigner au plus de disques de rayon $1/2$ et $\sqrt{2}/2$, on ne peut suivre le bord de nos disques que par l'intermédiaire des voisins 4-connexes et 8-connexes de chaque point du bord.

\paragraph{}
Prendre un $\alpha$ < -2 conserve un sens. Il représente le cas où l'$\alpha$-hull est constitué de l'ensemble disjoint des points appartenant au disque. L'$\alpha$-shape n'a pas d'existence propre dans ce cas au vue la non-connexité de l'ensemble.

\paragraph{}
On ne peut aller que jusqu'à un certain rayon $\alpha = 1/R_D$. Au delà, on aurait $R_{\alpha} < R_D$ et il serait alors impossible de récupérer l'ensemble des points de notre disque par l'intersection de disque avec un rayon inférieur.

\paragraph{}
Une propriété \cite{EdeKirSei83} des de l'union des $\alpha$-shape pour le cas négatif et le cas positif est d'obtenir des sous-ensembles des triangulations d'ordre 0 et d'ordre n de Delaunay.  
% need more 

%-----------------------------------------------------------------
\subsection{Triangulation de Delaunay}
%-----------------------------------------------------------------
% à refaire

Tout ensemble discret peut s'ecrire comme un ensemble dénombrable de points. Le construction d'une surface à partir de cet ensemble de points est souvent réaliser par l'étude des diagrammes de proximité : les diagrammes de Voronoï de l'ensemble de points. on construit implicitement leurs graphes duales, les triangulations de Delaunay.


%-----------------------------------------------------------------
\subsubsection{Diagramme de Voronoï}

Les diagrammes de Voronoï sont nommés d'après le mathématicien russe Georgi Voronoï : 1868-1908.

\begin{Definition}{Diagramme de Voronoï d'ordre 0 d'un ensemble de point $\mathcal{S}$}\\
\label{def:tri-vor-0}
  Le diagramme de Voronoï d'ordre 0 réalise une partition dans $\mathbb{R}^2$ où chaque point de $\mathcal{S}$ est inclue dans une cellule convexe contenant les points de $\mathbb{R}^2$ plus proches de ce point que de tous les autres selon une certaine distance.
\end{Definition}

 \begin{Definition}{Diagramme de Voronoï d'ordre n de l'ensemble $\mathcal{S}$}\\
\label{def:tri-vor-n}
   Le diagramme de Voronoï d'ordre n réalise une partition dans $\mathbb{R}^2$ où chaque point de $\mathcal{S}$ est associé à une cellule convexe contenant les points de $\mathbb{R}^2$ plus loin de lui que de n'importe quels autres points selon une certaine distance.
\end{Definition}

\begin{figure}[h!]
  \centering
 % \includegraphics[width=0.8\linewidth,page=2]{fig/2-mot/tri/mot-voronoi.pdf}
  \caption{Diagramme de Voronoï d'ordre 0 et d'ordre n}
\end{figure}

Leurs graphes duales, les triangulations de Delaunay possèdent des propriétés intéressantes.  Elles ont été nommées d'après le mathématicien russe Boris Delone : 1890-1980.

\begin{Definition}{Triangulation de Delaunay d'ordre 0 de l'ensemble $\mathcal{S}$}\\
\label{def:tri-del-0}
  La triangulation de Delaunay d'ordre 0 est une triangulation ou chaque disque circonscrit au triangle ne contient aucun autre point que les sommets du triangle.
\end{Definition}

\begin{Definition}{Triangulation de Delaunay d'ordre n de l'ensemble $\mathcal{S}$}\\
\label{def:tri-del-n}
  La triangulation de Delaunay d'ordre n est une triangulation ou chaque disque circonscrit au triangle contient tous les points de l'ensemble.
\end{Definition}

\begin{figure}[h!]
  \centering
 % \includegraphics[width=0.4\linewidth,page=2]{fig/2-mot/tri/mot-delaunay.pdf}
  \caption{Triangulation de Delaunay d'ordre 0 et d'ordre n}
\end{figure}
