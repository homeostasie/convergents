%%%%%%%%%%%%%%%%%%%%% chapter.tex %%%%%%%%%%%%%%%%%%%%%%%%%%%%%%%%%
%
% sample chapter
%
% Use this file as a template for your own input.
%
%%%%%%%%%%%%%%%%%%%%%%%% Springer-Verlag %%%%%%%%%%%%%%%%%%%%%%%%%%

\chapter{Suivi de bord}
\label{pt3-ch1-sb} % Always give a unique label
% use \chaptermark{}
% to alter or adjust the chapter heading in the running head

\section{$\alpha = 2$ et $\alpha = \sqrt{2}$}
\label{pt3-ch1-sec:1}

\subsection{Principe et méthodes}
\label{pt3-ch1-sec:1-1}

La première étape consiste à trouver un point de départ pertinent pour notre algorithme. Il a été choisi commencer avec le point le plus en bas à droite à l'intérieur de notre cercle. En effet, suite à un mauvais départ l’algorithme peut très bien ne pas aboutir ou plus génant encore aboutir à un résultat faux. \\

(?? Explication vulgaire de la méthode qui retourne le premier point à partir du centre du cercle et de la taille du rayon. ??) \\

Une fois notre premier point trouvé, l'algorithme va se poursuivre en traquant les points également sur le bord en suivant une certaine la direction de l'extérieur. Dans notre cas, il a été choisir de parcourir le cercle dans le sens trigonométrique. La première direction suivi est donc celle du vecteur $\overrightarrow{(1,0)}$.\\

% graph des possibilités de tracking
\begin{figure}[h!]
\centering
   \includegraphics[height=2cm]{pics/1-1_geodiscete.png}
\caption{Les possibilités d'évolution dans le cas $\alpha = 2$ puis $\alpha = \sqrt{2}$}
\end{figure}

Deux valeurs de $\alpha$ donne ici des enveloppes particulièrement intéressantes. On peut distinguer le cas $\alpha = 2$ et le cas $\alpha = \sqrt{2}$. Le premier cas permet de traquer, d'aller d'un point à un autre par 8 chemins possibles alors que le deuxième cas ne permet de se déplacer du point à son suivant que par seulement quatre chemins.\\


\section{Résultats}
\label{pt3-ch1-sec:2}

% quelques exemples des deux cas
\begin{figure}[h!]
\centering
   \includegraphics[height=6cm]{pics/1-1_geodiscete.png}
\caption{Cercles discrèts pour $\alpha = 2$ et $\alpha = \sqrt{2}$.}
\end{figure}


% graph du temps/ nombre de sommets/ taille de rayon
\begin{figure}[h!]
\centering
   \includegraphics[height=4cm]{pics/1-1_geodiscete.png}
\caption{Évolution du nombre de sommets et du temps en fonction de la taille du rayon pour le cas $\alpha = 2$. (Échelle logarithme)}
\end{figure}


\begin{figure}[h!]
\centering
   \includegraphics[height=4cm]{pics/1-1_geodiscete.png}
\caption{Évolution du nombre de sommets et du temps en fonction de la taille du rayon pour le cas $\alpha = \sqrt{2}$. (Échelle logarithme)}
\end{figure}


% tableau de la convergence
\begin{tabular}{| l || c | r | }
 \hline                 
   1 & 2 & 3 \\
   4 & 5 & 6 \\
   7 & 8 & 9 \\
 \hline  
 \end{tabular}
 
 
(?? // mettre le tableau sur le côté, On continue un peu la discussion par ici ??)

\chapter{$\alpha = 0$ - Enveloppe convexe}
\label{pt3-ch1--ch}

\begin{theorem}{Enveloppe Convexe}\newline
La définition mathématiques de l'enveloppe convexe passe par ici.
\end{theorem}

\section{Explicite - Graham}
\label{pt3-ch2-sec:1}

Algorithme assez connu qui consiste à prendre comme sommet suivant de l'enveloppe convexe le point formant une arête d'angle maximal avec l'arête précédente. Une solution pour connaitre l'orientation d'un triangle formé par trois points est de calculé l'aire algébrique. Pour rester cohérent avec les contraintes informatiques, il a été choisi de calculer l'aire du parallélogramme par la formule suivante.

\begin{theorem}{Aire du parallèlogramme formé par les trois points a, b, c}\newline
$$\mathnormal{A} = a[0]*(b[1] - c[1]) - b[0]*(a[1] - c[1]) + c[0]*(a[1] - b[1]){}$$
\end{theorem}

\begin{figure}[h!]
\centering
   \includegraphics[height=2cm]{pics/1-1_geodiscete.png}
\caption{Orientation d'un triangle formé par trois points.}
\end{figure}

\section{Implicite - Har-Peled}
\label{pt3-ch2-sec:2}

Le travail effectué dans \cite{HarPeled98} a été le point de départ de mon stage. Dans un premier temps, j'ai cherché à comprendre la notion de convergent, de jeter de rayon en implémentant le calcul du pgcd par la méthode géométrique. (p358) \newline
À partir d'une publication : \cite{HarPeled98} traitant d'un algorithme output sensitive pour calculer l'enveloppe convexe en $O(n^{2/3})$. -- maitre le O de la complexité...\newline

--> Intérêt d'une complexité faible, Étendre cette méthode de calcul géométrique à d'autre objet que l'enveloppe convexe.

(?? L'idée devrait également de pouvoir ajouter la version algorithm2e de l'algo, ou bien de la mettre en annexe ??)

\section{Résultats}
\label{pt3-ch1-sec:2}

% quelques exemples des deux cas
\begin{figure}[h!]
\centering
   \includegraphics[height=6cm]{pics/1-1_geodiscete.png}
\caption{Cercles discrèts.}
\end{figure}


% graph du temps/ nombre de sommets/ taille de rayon
\begin{figure}[h!]
\centering
   \includegraphics[height=4cm]{pics/1-1_geodiscete.png}
\caption{Évolution du nombre de sommets et du temps en fonction de la taille du rayon. (Échelle logarithme)}
\end{figure}


% tableau de la convergence
\begin{tabular}{| l || c | r | }
 \hline                 
   1 & 2 & 3 \\
   4 & 5 & 6 \\
   7 & 8 & 9 \\
 \hline  
 \end{tabular}


