%%%%%%%%%%%%%%%%%%%%% chapter.tex %%%%%%%%%%%%%%%%%%%%%%%%%%%%%%%%%
%
% sample chapter
%
% Use this file as a template for your own input.
%
%%%%%%%%%%%%%%%%%%%%%%%% Springer-Verlag %%%%%%%%%%%%%%%%%%%%%%%%%%

\chapter{$\alpha \leq 0$ - Généralisation de Har-Peled}
\label{pt4-ch1-leq0} % Always give a unique label

\section{Introduction}
\label{pt4-ch1-sec:1}

\begin{theorem}{$\alpha$-hull pour $\alpha \leq 0$}\newline
the alpha-shape is defined as the intersection of all closed complements of discs with radius -1/alpha that contain all the points inside the circle.
\end{theorem}

\section{Algorithme}
\label{pt4-ch1-sec:2}


\section{Résultats}
\label{pt4-ch1-sec:3}

\chapter{$\alpha \geq 0$}
\label{pt4-ch2-geq0} % Always give a unique label

\section{Introduction}
\label{pt4-ch2-sec:1}

\begin{theorem}{$\alpha$-hull pour $\alpha \geq 0$}\newline
the alpha-shape is defined as the intersection of all the closed discs with radius 1/alpha that contain all the points inside the circle.
\end{theorem}

\section{Algorithme}
\label{pt4-ch2-sec:2}


\section{Résultats}
\label{pt4-ch2-sec:3}


