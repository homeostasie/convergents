%------------------------------------------------
\section{Méthodes de calculs existantes}
%------------------------------------------------

%-----------------------------------------------------------------
\subsection{Suivi de Bord}
%-----------------------------------------------------------------

%-----------------------------------------------------------------
\subsection{Enveloppe Convexe}
%-----------------------------------------------------------------

Le calcul d'une $\alpha$-shape pour $\alpha$ = 0 revient à calculer l'enveloppe convexe de notre disque discret. Calculer une enveloppe convexe est particulièrement étudié. De nombreux algorithmes existes : Parcours de Graham, Marche de Jarvis... 

\begin{Definition}{$\alpha$-hull, $\alpha = 0$ - Enveloppe convexe}\\
\label{def:ch}
      L'intersection de tous les demi-plans qui contiennent tous les points à l’intérieur du disque.
\end{Definition}

Une méthode particulièrement performante à néanmoins retenue notre attention. Elle a été introduite par Har-Peled en 98 dans la publication suivante : \cite{HarPeled98}. Il y présente un algorithme de calcul de l'enveloppe convexe qui est incrémental et output sensitive.

\begin{Definition}{Méthode incrémentale}\\
\label{def:os}
      L'aspect incrémental de l'algorithme permet de découvrir les éléments succéssivements. 
\end{Definition}
Cela se traduit dans notre cas par la possibilité de trouver le prochain sommet de l'enveloppe à partir de n'importe quel sommet.  

\begin{Definition}{Output sensitive}\\
\label{def:os}
      Un algorithme output sensitive possède un temps d’exécution qui dépend de la taille de la sortie.
\end{Definition}

Dans ce cas de figure bien précis, la méthode présentée dépend du nombre de sommets de notre enveloppe convexe. \\

La méthode permet de construire successivement et rapidement les arêtes de notre polygone. Elle s'appuie sur le calcul de convergents, qui représente le pendant géométrique du calcul du pgcd de deux nombres entiers. 
En effet, il a été démontré dans ce même article que la compléxité en temps de cette algorithme de cacul de l'envelope convexe d'un disque de rayon $R_D$ révèle d'une part de la recherche du prochain sommet en $O(log R_D)$ et également du nombre de sommet qui est $O( R_{D}^{2/3})$. Soit une compléxité totale en temps de  $O( R_{D}^{2/3} log R_D)$.


%-----------------------------------------------------------------
\subsubsection{Calcul des convergents}

Cette méthode de calcul est géométrique. On transforme notre deux nombres entiers $x, y$ en un point de $\mathbb{Z}^{2}$ qu'on appellera $P = (P_x, P_y)$. À partir de l’origine $O=(0,0)$, on va chercher à trouver le premier point appartenant au segment de droite [O,P]. Le coefficient trouvé correspondant finalement au pgcd de $P_x$ et $P_y$.\\

Pour cela on met en place une méthode récursive.\\
Soient $p_{-2} = (1,0)$ et $p_{-1} = (0,1)$ les deux premiers convergents.\\

Pour trouver les convergents suivants, on va chercher :
$$p_{k} = p_{k-2} + q_k * p_{k-1}$$

avec $q_k$ le plus grand entier tel que $p_{k}$ et $p_{k-2}$ soient du même côté de la droite.\\

On se retrouve donc à effectuer un jeter de rayon de $p_{k-2}$ dans la direction de $p_{k-1}$ pour étudier l’intersection du vecteur et de la droite de direction $y_P / x_P$.\\

La méthode s’arrêtant quand enfin un convergent $p_{k}$ est exactement sur la droite.

%relié le tout à la fraction continue et revenir un peu à la méthode d'Euclide.

%-----------------------------------------------------------------
\subsubsection{Passage au disque}

La première étape de l'algorithme consiste à trouver un premier point effectivement sommet de l'enveloppe convexe. Plusieurs moyens sont possible. Nous avons choisit de trouver le point le plus en bas puis le plus à droite de tous les points du disque en cherchant à proximité d'un point éloigné du centre vers la bas d'une distance égale à celle du rayon.\\ 

%dire comment

Pour la suite, nous avons procéder de la même manière en construisant les convergents au plus prêt du bord du disque. En effet, nous allons successivement être à l'intérieur pour $k$ impaire et à l'extérieur du disque pour $k$ pair. Nous nous arrêterons quand le convergent se situe exactement sur le bord du disque ou bien au dernier convergent de degré impaire quand nous aboutirons à un rayon qui n'intersecte pas le disque.\\

Ainsi, nous récupérons successivement les sommets de l'enveloppe convexe jusqu'à retomber sur le premier sommet fermant ainsi notre enveloppe.

%pics pour rajouter du cachet

\subsubsection{Résultats}


On présente ici nos résultats obtenus qui sont conformes à ceux de la publication \cite{HarPeled98}. La procédure utilisée a été sensiblement la même en simulant pour chaque rayon augmentant d'un facteur deux, cent disques dont le centre est compris dans le carré $[0,1]\times[0,1]$.

On s'intéresse également à vérifié l'intérêt de notre algorithme par rapport à la marche de Graham qui nous a permit de vérifier la justesse de nos calculs. 


\begin{tabular}{|l||c|c|}
\hline
Rayons & Nb Sommets & Nb Sommets / $R^{2/3}$\\
\hline

\hline
\end{tabular} 

\begin{tabular}{|l||c|c|}
\hline
Rayons & Nb Sommets & Nb Sommets / $R^{2/3}$\\
\hline

\hline
\end{tabular} 

%-----------------------------------------------------------------
\subsection{Triangulation de Delaunay d'un motif}
%-----------------------------------------------------------------


La construction de notre enveloppe convexe nous a amené à construire ses arêtes. Nous avons ainsi construit une série de segment de droite englobant tous les points à l'intérieur de notre disque. De manière encore plus fines, nous avons construit une série de motif associé à chaque segment de droite que nous pouvons relié à leur triangulation de Delaunay. Or, les $\alpha$-shapes s’appuient également sur l'utilisation de la triangulation de Delaunay, d'ordre 0 lorsque $\alpha \leq 0$ et d'ordre n lorsque $\alpha \geq 0$. On a également vu qu'il était possible d'utiliser un algorithme "Output Sensitive" pour le calcul de l'enveloppe convexe.

La problématique de ce stage est de chercher à trouver s'il est possible de calculer les $\alpha$-shapes de manière output sensitive en s'appuyant sur le calcul des convergents et par la suite d'implémenter un tel algorithme.  

