%%%%%%%%%%%%%%%%%%%%% chapter.tex %%%%%%%%%%%%%%%%%%%%%%%%%%%%%%%%%
%
% sample chapter
%
% Use this file as a template for your own input.
%
%%%%%%%%%%%%%%%%%%%%%%%% Springer-Verlag %%%%%%%%%%%%%%%%%%%%%%%%%%

\chapter{Conclusion}
\label{pt5-ch1-con} % Always give a unique label
% use \chaptermark{}
% to alter or adjust the chapter heading in the running head

% En premier lieu il faut faire une transition qui permettra de passer en douceur du développement à la conclusion ; il convient ensuite de passer à un résumé du développement, et enfin de faire une ouverture. La conclusion va du particulier vers le général, c’est-à-dire qu’elle suit le processus inverse de celui adopté dans l’introduction. 

%%%%%%%%%%%%%%%%%%%%%%%%%%%%%%%%%%%%%%%%%%%%%%%%%%%%%%%%%%%%%%%%%%%

\section{Développement}
\label{pt5-ch1-1}

\subsection{Transition}
\label{pt5-ch1-sec:1.1}

\subsection{Objectif}
\label{pt5-ch1-sec:1.2}

\section{Ouverture}
\label{pt5-ch1-sec:2}

