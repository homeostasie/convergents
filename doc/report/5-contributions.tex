%------------------------------------------------
\section{Contributions}
%------------------------------------------------

%-----------------------------------------------------------------
\subsection{$\alpha$-shape, $\alpha \leq 0$ - Généralisation de Har-Peled}
%-----------------------------------------------------------------

%-----------------------------------------------------------------
\subsubsection{Construction de l'algorithme}

Le principe de l’algorithme est assez simple. On cherche à reproduire le schéma du calcul de l’enveloppe convexe. Néanmoins, à chaque convergents à l'intérieur du disque on va contrôler s'il est possible pour lui d'être atteint par un le rayon d'un disque généralisé.

%-----------------------------------------------------------------
\subsubsection{Résultats}

Les résultats présentés sont : 

\begin{tabular}{|l|c||c|c|}
\hline
Rayons & Prédicat & Nb Sommets & Nb Sommets / $R^{2/3}$\\
\hline

\hline
\end{tabular} 



%-----------------------------------------------------------------
\subsection{$\alpha$-shape, $\alpha \geq 0$}
%-----------------------------------------------------------------

%-----------------------------------------------------------------
\subsubsection{Construction de l'algorithme}


Notre version de l'algorithme s'appuie ici sur les sommets de l'enveloppe convexe. En effet, on va chercher à savoir quel sous-ensemble de point compose notre $\alpha$-shape.

%-----------------------------------------------------------------
\subsubsection{Résultats}

Les résultats présentés sont : 

\begin{tabular}{|l|c||c|c|}
\hline
Rayons & Prédicat & Nb Sommets & Nb Sommets / $R^{2/3}$\\
\hline

\hline
\end{tabular} 


