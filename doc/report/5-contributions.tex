%------------------------------------------------
\section{Contributions}
%------------------------------------------------

%-----------------------------------------------------------------
\subsection{$\alpha$-shape, $\alpha \leq 0$ - Généralisation de Har-Peled}
%-----------------------------------------------------------------

\subsubsection{Présentation}

Un disque fermé généralisé avec un rayon inférieur à zéro correspond au complémentaire fermé d'un disque. La définition peut alors se réécrire de la façon suivante.
\begin{Definition}{$\alpha$-hull, $\alpha < 0$}\\
\label{def:nas}
    L'intersection de tous les complémentaires fermés de disques de rayon -1/$\alpha$ qui contiennent tous les points à l’intérieur du disque.
\end{Definition}

Pour $\alpha = -2$, on se retrouve avec le plus petit rayon possible pour le disque généralisé. En effet, le rayon du disque $R_{\alpha} = 1/2$ possède un diamètre de 1 et permet donc à un point d'uniquement rejoindre son voisin 4-connexes situé à une unité de lui. Prendre un $\alpha$ plus petit conserve un sens. Il représente le cas où l'$\alpha$-hull est consitué de l'ensemble disjoint des points appartenant au disque. L'$\alpha$-shape n'a pas d'existence propre dans ce cas au vue de l'ensemble pas connexe. On se concentrera donc à des $\alpha \leq -2$ qui iront jusqu'à $\alpha = -\epsilon$ aussi petit que possible par des valeurs négatives pour s'approcher de l'enveloppe convexe en prenant néanmoins les point qui sont sur le bord de l'enveloppe convexe mais qui ne sont pas arêtes.


\subsubsection{Construction de l'algorithme}

Le principe de l’algorithme est assez simple. On cherche à reproduire le shéma du calcul de l’enveloppe convexe. Nénamoins, à chque convergents à l'intérieur du disque on va contrôler s'il est possible pour lui d'être atteinds par un le rayon d'un disque généralisé.


\subsubsection{Résultats}

Les résultats présentés sont : 

\begin{tabular}{|l|c||c|c|}
\hline
Rayons & Prédicat & Nb Sommets & Nb Sommets / $R^{2/3}$\\
\hline

\hline
\end{tabular} 



%-----------------------------------------------------------------
\subsection{$\alpha$-shape, $\alpha \geq 0$}
%-----------------------------------------------------------------

\subsubsection{Présentation}

\begin{Definition}{$\alpha$-hull, $\alpha > 0$}
\label{def:pas}
      L'intersection de tous les disques fermés de rayon 1/$\alpha$ qui contiennent tous les points à l’intérieur du disque.
\end{Definition}

On ne peut aller que jusqu'à un certain rayon $\alpha = 1/R_D$. Au delà, on aurait $R_{\alpha} < R_D$ et il serait alors impossible de récupérer l'ensemble des points de notre disque par l'intersection de disque avec un rayon inférieur. On remarque également que 
pour $\alpha = -\epsilon$ aussi petit que possible par des valeurs positive on retrouve éxactement le calculd de l'enveloppe convexe.
\subsubsection{Construction de l'algorithme}

Notre version de l'algorithme s'appuie ici sur les sommets de l'enveloppe convexe. En effet, on va chercher à savoir quel sous-ensemble de point compose notre $\alpha$-shape.

\subsubsection{Résultats}

Les résultats présentés sont : 

\begin{tabular}{|l|c||c|c|}
\hline
Rayons & Prédicat & Nb Sommets & Nb Sommets / $R^{2/3}$\\
\hline

\hline
\end{tabular} 


