%------------------------------------------------
\section{Contributions}
%------------------------------------------------

%-----------------------------------------------------------------
\subsection{Généralisation de Har-Peled - $\alpha \leq 0$}
%-----------------------------------------------------------------

\subsubsection{Présentation}

Un disque fermé généralisé avec un rayon inférieur à zéro correspond au complémentaire fermé d'un disque. La définition peut alors se réécrire de la façon suivante.
\begin{Definition}{$\alpha$-hull, $\alpha < 0$}
\label{def:nas}
    L'intersection de tous les complémentaires fermés de disques de rayon -1/$\alpha$ qui contiennent tous les points à l’intérieur du disque.
\end{Definition}

Pour $\alpha = -2$, on se retrouve avec le plus petit rayon possible pour le disque généralisé. En effet, le rayon du disque $R_{\alpha} = 1/2$ possède un diamètre de 1 et permet donc à un point d'uniquement rejoindre son voisin 4-connexes situé à une unité de lui. Prendre un $\alpha$ plus petit conserve un sens. Il représente le cas où l'$\alpha$-hull est consitué de l'ensemble disjoint des points appartenant au disque. L'$\alpha$-shape n'a pas d'existence propre dans ce cas au vue de l'ensemble pas connexe. On se concentrera donc à des $\alpha \leq -2$ qui iront jusqu'à $\alpha = -\epsilon$ aussi petit que possible par des valeurs négatives pour s'approcher de l'enveloppe convexe en prenant néanmoins les point qui sont sur le bord de l'enveloppe convexe mais qui ne sont pas arêtes.


\subsubsection{Construction de l'algorithme}

\subsubsection{Résultats}





%-----------------------------------------------------------------
\subsection{$\alpha \geq 0$}
%-----------------------------------------------------------------

\subsubsection{Présentation}

\subsubsection{Construction de l'algorithme}

\subsubsection{Résultats}
\begin{Definition}{$\alpha$-hull, $\alpha > 0$}
\label{def:pas}
      L'intersection de tous les disques fermés de rayon 1/$\alpha$ qui contiennent tous les points à l’intérieur du disque.
\end{Definition}




