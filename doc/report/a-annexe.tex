%------------------------------------------------
\section{Annexes}
%------------------------------------------------

%-----------------------------------------------------------------
\subsection{Équivalence entre le calcul du pgcd par la méthode Euclidienne et géométrique}
\label{annexe-euc-geo}
%-----------------------------------------------------------------

%-----------------------------------------------------------------
\subsubsection{Trouver $q_n$}

Soit d la droite d'équation ax + by + c = 0 et soit un rayon issue de $S=(x_s, y_s)$ dans la direction du vecteur $V=(x_v, y_v)$.\\
Soit $M(x,y)$ un point tel que $\overrightarrow{SM} = q \overrightarrow{V}$. Les coordonnées de M vérifient : \\
 
\begin{equation*}
    \left\{
    \begin{split}
      x - x_s &= q x_v \\
      y - y_s &= q y_v 
   \end{split}
   \right. 
\end{equation*}

\begin{equation*}
    \left\{
    \begin{split}
      x &= q x_v + x_s \\
      y &= q y_v + y_s 
   \end{split}
   \right. 
\end{equation*}


On choisit le point M comme l'intersection de la droite d d'équation d : ax + by +c = 0 et du rayon. Les coordonnées de M vérifient : \\

$$(a x_v + b y_v) * q = -(a x_s + b y_s +c)$$
$$q = - \frac{a x_s + b y_s +c}{(a x_v + b y_v)}$$

%-----------------------------------------------------------------
\subsubsection{Rappel sur le calcul du pgcd par l'algorithme d'Euclide}


On rappelle les résultats de l'algorithme d'Euclide pour a,b : \\
\noindent $a       = k_0*b       + r_0$ d'où $k_0 = \lfloor b / a \rfloor$\\
\noindent $b       = k_1*r_0     + r_1$\\
\noindent $r_{n-2} = k_n*r_{n-1} + r_{n}$\\

%-----------------------------------------------------------------
\subsubsection{Preuve par récurrence sur n que $q_n = k_n$}

\paragraph{}

On vérifie que que la $q_0 = k_0$.\\

Soit de la droite passant par O(0,0) et P(b, a) d'équation d: ax + by =0 et soit le rayon issue de $p_{-2} = (1,0)$ dans la direction de $p_{-1} = (0,1)$.\\

On note le convergent $p_{0} = p_{-2} + q_0 p_{-1}$ avec $q_0$ est le plus grand entier tel que $p_0$ et $p_{-2}$ soient du même côté. Sa valeur est la partie entière de $-(a x_s + b y_s +c)/(a x_v + b y_v)$

$$q_0 = - \lfloor\frac{a*1 + b*0 +0}{a*0 + b*1}\rfloor = - \lfloor\frac{a}{b}\rfloor = k_0$$

On a montré que la relation est vraie au rang 0 avec \textbf{$q_0 = k_0$}.

\paragraph{}
On suppose vrai la relation au rang n.

$q_n = k_n$\\

Ce qui implique :  
$r_{n-2} = k_n*r_{n-1} + r_{n}$\\
$a x_{n-2} + b y_{n-2} = r_{n-2}$\\
$a x_{n-1} + b y_{n-1} = r_{n-1}$\\

Soit le rayon issue de $p_{n-1}$ dans la direction de $p_{n}$.\\
On note le convergent $p_{n+1} = p_{n-1} + q_{n+1} p_{n}$. $q_{n+1}$ est le plus grand entier tel que $p_{n+1}$ et $p_{n-1}$ soient du même côté. Sa valeur est la partie entière de $-(a x_{n-1} + b y_{n-1} +c)/(a x_{n} + b y_{n})$


$q_0$ est égale au premier coefficient de la division Euclidienne de b par a.

\begin{align*}
q_{n+1} &= - \lfloor\frac{a x_{n-1} + b y_{n-1}}{ a x_{n} + b y_{n}}\rfloor \\
        &= - \lfloor\frac{a x_{n-1} + b y_{n-1}}{ (a (x_{n-2} + q_n * x_{n-1}) + b (y_{n-2} + q_n * y_{n-1})}\rfloor \\
        &= - \lfloor\frac{a x_{n-1} + b y_{n-1}}{ q_n*(a x_{n-1} + b*y_{n-1} ) + a x_{n-2} + b*y_{n-2}  }\rfloor \\
        &= - \lfloor\frac{r_{n-1}              }{ r_{n-2} + k_n * r_{n-1}}\rfloor \\
        &= - \lfloor\frac{r_{n-1}              }{ r_{n} }\rfloor \\
        &= k_{n+1}
\end{align*}

On a montré que \textbf{$q_{n+1}= k_{n+1}$}. Comme la relation est vraie au rang 0 et au rang (n+1),  \textbf{elle est vraie pour tout n > 0}.

%-----------------------------------------------------------------
%%THOM
\subsubsection{Exemple pour le calcul du pgcd et des convergents de (3,8)}

%% TRI: je ne vois pas l'intérêt ici: en annexe ?
 \begin{table}[H]
   \centering
   \begin{tabular}{|p{0.04\linewidth}|p{0.04\linewidth}|p{0.04\linewidth}||p{0.04\linewidth}|p{0.04\linewidth}||p{0.2\linewidth}||p{0.35\linewidth}|}
     \hline 
     $k$ & $a$ & $b$ & $q_{k}$ & $r_k$ & (a,b) & $p_k$ \\
     \hline 
     -2 & - & - & - & - & -                    & $p_{-2} = (1,0)$\\
     -1 & - & - & - & - & $3 p_{-2} + 8p_{-1}$ & $p_{-1} = (0,1)$\\
      0 & 8 & 3 & 2 & 2 & $2 p_{-1} + 3p_{0}$  & $p_{0}  = p_{-2} + 2p_{-1} = (1,2)$\\ 
      1 & 3 & 2 & 1 & 1 & $p_{0}  + 2p_{1}$    & $p_{1}  = p_{-1} + p_{0} = (1,3)$\\
      2 & 2 & 1 & 2 & 0 & $p_{2}$              & $p_{2}  = p_{0} + 2p_{1} = (3,8)$\\
     \hline
   \end{tabular} 
   \caption{Résultat du $PGCD(3,8)$ par la méthode d'Euclide et géométrique}
 \end{table}

%-----------------------------------------------------------------
\subsection{Algorithme du calcul de l'$\alpha$-shape pour $\alpha <0$}
%-----------------------------------------------------------------



\begin{algorithm}[h]
  \label{algo:1}
  \KwIn{$\alpha$, $v_{in}$ any $\alpha$-shape vertex  }
  \KwOut{$L$ a list of consecutive $\alpha$-shape vertices starting from $v_{in}$ and $v_{out}$ the last vertex.}
  %
  $p_{-2} \leftarrow (1,0)$, $p_{-1} \leftarrow (0,1)$, $k \leftarrow 0$, stop $\leftarrow$ false \;
  \While{Ray-shooting intersecte the circle and non stop} 
  {
    $p_k = q_k p_{k-1} + q_{k-2}$ \;
    \eIf{$p_k$ is outside the shape} 
    {
      \tcc{k is odd}	
      \If{$k > 0$ et $qk < 0$} 
      {
        $v_{out} \leftarrow p_{k-1}$ \;
        $stop\leftarrow true$ \;
      }
    }
    {
      \tcc{$p_k$ is inside or lie on the shape}	
      \eIf{$k\: is\: even$} 
      {
        \tcc{$R$ returns the circumcircle radius of three points}	
        \If{ $R(v, (q_k - 1)p_{k-1} + p_{k-2}, p_k) > -1/\alpha$ }
        {
           \tcc{Find the greatest integer $0 \leq q \leq q_{k} - 1$ such that}
          $R(v, qp_{k-1} + p_{k-2}, (q+1)p_{k-1} + p_{k-2}) > -1/\alpha$ \;
          
          $L \leftarrow L \cup v$ \; 
          \For{$i$ from $1$ to $q_k-q$ } 
          {
            $L \leftarrow L \cup i p_{k-1} $ \;      
          }
          $v_{out} \leftarrow p_{k}$ \;
          $stop \leftarrow true$ \;
        }          
      }  
      {
        \If{ $R(v, p_k, (q_k - 1)p_{k-1} + p_{k-2}) > -1/\alpha$ }
        {
           \tcc{Find the greatest integer $0 \leq q \leq q_{k} - 1$ such that}
          $R(v, qp_{k-1} + p_{k-2}, (q+1)p_{k-1} + p_{k-2}) > -1/\alpha$ \;
        
          \eIf{ $qkalpha == 0$} 
          {
            $v_{out} \leftarrow p_{k-2}$ \;
            $stop \leftarrow true$ \;
          }
          {
            \For{$i$ from $1$ to $q_k-q$ } 
            {
              $L \leftarrow L \cup p_{k-2} + i p_{k-1} $ \;      
            }
            $v_{out} \leftarrow p_{k}$ \;
            $stop \leftarrow true$ \;
          }  
        }
      } 
    }
    \tcc{Update  $p_{k-1}$ and  $p_{k-2}$}	    
    $k \leftarrow k + 1$ \;
    $p_{k-2} \leftarrow p_{k-1}$ \;
    $p_{k-1} \leftarrow p_{k}$ \;
   }
   \If{$stop == false$ }
   {
    $v_{out} \leftarrow p_{k-1}$ \; 
   } 
\end{algorithm}

